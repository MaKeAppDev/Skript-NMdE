%+------------------------------------------------------------------------------+
%| Numerische Methoden der Elektrotechnik                                       |
%| Skript geschrieben von Markus Hofbauer, Kevin Meyer und Benedikt Schmidt	    |
%+------------------------------------------------------------------------------+
\documentclass[fontsize=11pt,   % Schriftgröße
    DIV=12,          % Seitenaufteilung
    BCOR=5mm,        % Bindekorrektur
    ngerman,         % für Umlaute, Silbentrennung etc.
    paper=a4,        % Papierformat
    twoside,         % zweiseitig
    titlepage,       % es wird eine Titelseite verwendet
    parskip=half,    % Abstand zwischen Absätzen (halbe Zeile)
    listof=totoc,    % Verzeichnisse im Inhaltsverzeichnis aufführen
    ]{scrbook}

%% Packages
\usepackage[utf8]{inputenc}
\usepackage[german]{babel}
\usepackage{scientific}
\usepackage{subcaption}
\usepackage[european]{circuitikz}
\usetikzlibrary{trees,positioning,arrows,calc,intersections,patterns}
\usepackage{enumitem}
\usepackage{wasysym}
\usepackage{mathtools}
\usepackage{hyperref}
\hypersetup{
    colorlinks,
    citecolor=black,
    filecolor=black,
    linkcolor=black,
    urlcolor=black,
    %linktocpage % Seitenzahlen anstatt Text im Inhaltsverzeichnis verlinken
}
\usepackage{todonotes}

% Seitenstiel
\usepackage[automark, headsepline]{scrpage2}
\pagestyle{scrheadings}
\automark[section]{chapter}
% Trennlinie unter Kopfzeile
\setheadsepline[text]{0.4pt}
% Nummerierung der Kapitel beeinflussen
% -1 keine Überschrift wird numeriert
% 0  Kapitelüberschriften werden numeriert
% 1  Kapitel- und Abschnittsüberschriften werden numeriert
% 2  Kapitel- bis Unterabschnittsüberschriften werden numeriert
% 3  Kapitel- bis Unterunterabschnittsüberschriften werden numeriert
% 4  Kapitel- bis Paragraphsüberschriften werden numeriert
% 5  alle Überschriften werden numeriert
\setcounter{secnumdepth}{2}
\setcounter{tocdepth}{2}
% Schusterjungen und Hurenkinder vermeiden
\clubpenalty = 10000 
\widowpenalty = 10000 
\displaywidowpenalty = 10000

\newcommand{\rang}[1]{\ensuremath{\text{rang}\left(#1\right)}}
\newcommand{\cond}[1]{\ensuremath{\text{cond}\left(#1\right)}}
\DeclareMathOperator{\log2}{ld}
\DeclareMathOperator{\col}{col}
\renewcommand{\ma}[1]{\ensuremath{\boldsymbol {#1}}}

\sisetup{per-mode=fraction, locale=DE}

\DeclareSIUnit \voltampere {VA}
\DeclareSIUnit \var {Var}
\DeclareSIUnit \newtonmeter {Nm}
\DeclareSIUnit \voltsecond {Vs}
\DeclareSIUnit \amperesecond {As}


\begin{document}
	
	\thispagestyle{plain}
\begin{titlepage}

\subject{\textsf{Mitschrift}}

\title{Numerische Methoden der Elektrotechnik}

\subtitle{TUM EI M.Sc. Wahlpflichtmodul}

\author{Markus Hofbauer \and Kevin Meyer \and Benedikt Schmidt}
\date{SS 2014\\[2cm]}

\publishers{
\begin{center}
\large\textbf{Dozent}\\Prof. Dr.-Ing. Ulf Schlichtmann\\
\end{center}}

\end{titlepage}
\maketitle
	\listoftodos
	% Inhaltsverzeichnis
	\cleardoublepage
	\pagenumbering{Roman}
	\pdfbookmark[1]{Inhaltsverzeichnis}{toc}
	\tableofcontents
	
	% Inhalt
	\cleardoublepage
	\pagenumbering{arabic}
	\chapter{Einführung}
\label{chap:einfuehrung}

\section{Zusammenfassung}

\subsection{Überblick}
\begin{center}
\begin{tikzpicture}[very thick,
		node distance=2.5cm,on grid,>=stealth',
		block/.style={rectangle,draw,rounded corners=1mm,minimum height=1cm,minimum width=2cm}]

\node [block] (top)						{Mathematik};
\node [block] (down3) [below=of top]	{Analysis} edge [-] (top);
\node [block] (down2) [left=of down3]	{Algebra} edge [-] (top);
\node [block] (down1) [left=of down2]	{\begin{tabular}{c}Logik und\\ Mengenlehre\end{tabular}} edge [-] (top);
\node [block] (down4) [right=of down3]	{Stochastik} edge [-] (top);
\node [block] (down5) [right=of down4]	{Numerik} edge [-] (top);
\end{tikzpicture}
\end{center}

\textbf{Definitionsversuch:} Entwicklung und mathematisches Verständnis von numerischen Algorithmen, als von Rechenmethoden zur zahlenmäßigen Lösung mathematischer Probleme.

\textbf{Zusammenspiel mit Informatik:}
\begin{center}
\begin{tikzpicture}[very thick,node distance=1.5cm,on grid,>=stealth']
\node (top)	{Numerische Mathematik};
\node (mid) [below=of top] {\begin{tabular}{l}Scientific Computing\\ Wissenschaftliches Rechnen\end{tabular}} edge [<-] (top);
\node (down) [below=of mid]	{Informatik} edge [<-] (mid);
\node [node distance=7cm, right=of mid] {\begin{tabular}{c}Logik und\\ Mengenlehre\end{tabular}} edge [<-] node [midway,above] {Computer} (mid);
\end{tikzpicture}
\end{center}

\subsection{Vorgehen}
\begin{center}
\begin{tikzpicture}[very thick,node distance=1.5cm,on grid,>=stealth']
\node (top)	{Anwendungsproblem};
\node (down1) [below=of top] {Primäres mathematisches Problem} edge [<-] node [right] {Mathematisierung} (top);
\node (down2) [below=of down1] {Sekundäre mathematische Probleme} edge [<-] node [right] {Mathematische Umformungen} (down1);
\node (down3) [below=of down2] {Algorithmus/Computerprogramm} edge [<-] node [right] {Programmierung} (down2);
\node (down4) [below=of down3] {"'Lösung"'} edge [<-] node [right] {Rechnung} (down3);
\node [below=of down4] {Lösung} edge [<-] node [right] {Beurteilung, "'Entmathematisierung"'} (down4);
\end{tikzpicture}
\end{center}

\subsection{Aspekte für die Lösung einer mathematischen Aufgabenstellung}
\begin{itemize}
\item Kondition eines Problems (Empfindlichkeit für Störungen)
\item Numerische Lösungsverfahren
\item Stabilität des Lösungsverfahrens (Empfindlichkeit für Störungen)
\item Effizienz des Lösungsverfahrens
\item Genauigkeit der Lösung
\end{itemize}

\section{Wiederholung}

\subsection{Lineares Gleichungssystem}

\begin{align*}
a_{11}\,x_1 + \ldots + a_{1n}\,x_n &= b_1\\
a_{21}\,x_1 + \ldots + a_{2n}\,x_n &= b_2\\
&\vdots\\
a_{m1}\,x_1 + \ldots + a_{mn}\,x_n &= b_m\\
\ma{A} \cdot \vec{x} &= \vec{b}\\
\ma{A} &\in \mathbb{R}^{m\times n},\ \vec{x} \in \mathbb{R}^n,\ \vec{b} \in \mathbb{R}^m
\end{align*}

\subsection{Matrizen}
\subsubsection{Elementare Matrix-Operationen}


\textbf{Addition:}
\begin{align*}
\ma{A}_{<m\times n>} + \ma{B}_{<m\times n>} &= \ma{C}_{<m\times n>}\\
c_{\mu\nu} &= a_{\mu\nu} + b_{\mu\nu}
\end{align*}

\textbf{Multiplikation mit Skalar:}
\begin{align*}
k \cdot \ma{A} = \ma{A} \cdot k = B\\
b_{\mu\nu} = k \cdot a_{\mu\nu}
\end{align*}

\textbf{Matrix-Multiplikation:}
\begin{align*}
\ma{A}_{<m\times n>} \cdot \ma{B}_{<n\times l>} &= \ma{C}_{<m\times l>}\\
c_{\mu\nu} &= \sum_{\nu=1}^{n} a_{\mu\nu} \cdot b_{\mu\lambda};\quad \mu=1\ldots m,\ \lambda=1\ldots l
\end{align*}

\textbf{Multiplikation Matrix-Vektor:}
\begin{align*}
\text{Spezialfall der Matrix-Multiplikation: } <n\times 1> \text{ bzw. } <1\times n>
\end{align*}

\textbf{Auffassen als Linearkombination der Spalten der Matrix:}
\begin{align*}
\ma{A} \cdot \vec{x} &= \vec{b}\\
\begin{bmatrix}\vec{a}_1 & \vec{a}_2 & \ldots & \vec{a}_n\end{bmatrix} \cdot \vec{x} &= \vec{b}\\
\vec{a}_1\,\vec{x} + \ldots + \vec{a}_n\,\vec{x}_n &= \vec{b}
\end{align*}

\textbf{Vektoren:}
\begin{align*}
\text{Spaltenvektor:} \quad & \vec{a}_{<n\times 1>}: \vec{a}\\
\text{Zeilenvektor:} \quad & \vec{b}_{<1\times m>}^T: \vec{b}^T
\end{align*}

\subsubsection{Vektormultiplikation}
\begin{align*}
\vec{b}^T\cdot \vec{a} &= c \quad \text{(Skalarprodukt $m=n$!)}\\
\vec{a} \cdot \vec{b}^T &= \ma{C} \quad \text{(Vektorprodukt, dyadisches Produkt)}
\end{align*}

\textbf{Matrix als Kombination von Zeilen- und Spaltenvektoren:}
\begin{align*}
\ma{A}_{<m\times n>} &= \begin{bmatrix} \vec{a}_1^T \\ \vec{a}_2^T \\ \vdots \\ \vec{a}_m^T \end{bmatrix}\\
\ma{B}_{<n\times l>} &= \begin{bmatrix} \vec{b}_1 & \vec{b}_2 & \ldots & \vec{b}_l \end{bmatrix}\\
\ma{A} \cdot \ma{B} &= \ma{C}\\
\vec{c}_{\mu\lambda} &= \vec{a}_\mu^T \cdot \vec{b}_\lambda
\end{align*}

\textbf{Rechenregeln:}
\begin{align*}
\left(\ma{A}\cdot \ma{B}\right)\cdot \ma{C} &= \ma{A} \cdot \left(\ma{B} \cdot \ma{C}\right) \quad \text{Assoziativität}\\
\ma{A} \cdot \left(\ma{B} + \ma{C}\right) &= \ma{A}\,\ma{B} + \ma{A}\,\ma{B} \quad \text{Distributivität}\\
\ma{A} \cdot \ma{B} &\neq \ma{B} \cdot \ma{A} \quad \text{Kumutativität gilt im Allgemeinen nicht}
\end{align*}

\textbf{Diagonalmatrix:}
\begin{align*}
\ma{D} &= \text{diag}\begin{bmatrix}d_1 & d_2 & \ldots & d_n\end{bmatrix} = \begin{bmatrix}d_1 & \ldots & 0 \\ 0 & \ldots & d_n\end{bmatrix}\\
\ma{D}_1 \cdot \ma{D}_2 &= \ma{D}_2 \cdot \ma{D}_1
\end{align*}

\textbf{Einheitsmatrix:}
\begin{align*}
\ma{I} = \ma{E} = \ma{1} = \text{diag}\begin{bmatrix}1 & \ldots & 1\end{bmatrix}\\
\ma{A} \cdot \ma{I} = \ma{I} \cdot \ma{A}\\
\ma{I}^n = \ma{I}\\
\ma{I}^{-1} = \ma{I}
\end{align*}

\textbf{Transponierte Matrix:}
\begin{align*}
\ma{A}_{<m\times n>}^T &= \ma{B}_{<n\times m>},\ b_{\nu\mu} = a_{\mu\nu}\\
\left(\ma{A} \cdot \ma{B}\right)^T &= \ma{B}^T \cdot \ma{A}^T\\
\ma{A} &= \ma{A}^T \quad \Rightarrow \text{ symmetrische Matrix}
\end{align*}

\textbf{Inverse Matrix:}
\begin{align*}
\ma{A} &\in \mathbb{R}^{n\times n}\\
\ma{A}^{-1} \cdot \ma{A} &= \ma{A} \cdot \ma{A}^{-1} = \ma{I}\\
\text{$\ma{A}^{-1}$ existiert nur für nicht singuläre $\ma{A}$}&\text{ und ist eindeutig.}\\
\ma{A} = \text{diag}\begin{bmatrix}d_1 & \ldots & d_n\end{bmatrix} & \Rightarrow \ma{A}^{-1} = \text{diag}\begin{bmatrix}\frac{1}{d_1} & \ldots & \frac{1}{d_n}\end{bmatrix}\\
\left(\ma{A} \cdot \ma{B}\right) &= \ma{B}^{-1} \cdot \ma{A}^{-1}
\end{align*}

\section{Schaltungsanalyse}
\begin{center}
\begin{circuitikz}
\draw
	(0,4)
	to [R=$Y_1$, i=\textcolor{green!70!blue}{$i_1$},-o] (4,4) node[above] {1}
	to [R=$Y_3$, i=\textcolor{green!70!blue}{$i_3$},v>=\textcolor{blue}{$u_3$},-o] (8,4) node[above] {3}
	to [R=$Y_5$, i<^=\textcolor{green!70!blue}{$i_5$},v=\textcolor{blue}{$u_5$},o-o] (12,4) node[above] {2} 
	to [short,i<=\textcolor{green!70!blue}{$i_2$}] (12,3)
	to [R=$Y_2$,v=\textcolor{blue}{$u_2$},*-*] (12,1)
	to (13.5,1)
	to [I,i_=$I_{02}$] (13.5,3) to (12,3)
	(12,1) to (12,0) to (8,0) node[ground] {}
	to [R=$Y_4$, i=\textcolor{green!70!blue}{$i_4$},v>=\textcolor{blue}{$u_4$},o-] (8,4) 
	(8,0) +(0.2,0) node[above] {0} to (0,0)
	to [V=$U_{01}$] (0,4)
	(4,4) 
	to [open,v=\textcolor{blue}{$u_1$}] (0,0);

\draw[color=red](8,0)
	to [open,v^>=\textcolor{red}{$u_{n1}$}] (4,4)
	(9,0)
	to [open,v>=\textcolor{red}{$u_{n3}$}] (9,4)
	(11,0)
	to [open,v^>=\textcolor{red}{$u_{n2}$}] (11,4);

\end{circuitikz}
\end{center}
\begin{tabular}{llll}
\textcolor{green!70!blue}{$\vec{i}_{<k>}$} & \textcolor{green!70!blue}{Kantenstromvektor} & \textcolor{blue}{$\vec{u}_{<k>}$} & \textcolor{blue}{Kantenspannungsvektor}\\
$\vec{i}_{0<k>}$ & Kantenquellenstromvektor & $\vec{u}_{0<k>}$ & Kantenquellenspannungsvektor\\
$\vec{i}_{n<k>}$ & Knotenquellenstromvektor & \textcolor{red}{$\vec{u}_{n<k>}$} & \textcolor{red}{Knotenquellenspannungsvektor}\\
\\
$\ma{A}_{<n\times k>}$ & Kontenmatrix, Knoteninzidenzmatrix & &\\
$\ma{Y}_{<k\times k>}$ & Kantenadmittanzmatrix & &\\
$\ma{Y}_{n<n\times n>}$ & Kontenadmittanzmatrix & &\\
\end{tabular}

\subsection{Gerichteter Graph der Schaltung}

\subsection{Inzidenz Matrix}
\begin{align*}
\ma{A}_{<m\times n>} = \begin{bmatrix}
1 & & & -1\\
 & 1 & & -1\\
-1 & & 1 & \\
 & & 1 & -1\\
 & -1 & 1 & 
\end{bmatrix}
\end{align*}
Summen der Spaltenvektoren $= \vec{0}$.\\
$\Rightarrow \ma{A}$ hat linear abhängige Spalten.

Rang der Matrix $\ma{A}$: $r = \rang{\ma{A}} = 3 = n-1$.\\
Dimension des Nullraums: Zahl der Spalten $-\ r = 1$ .\\
Vektor im Nullraum von $\ma{A}$:
\begin{align*}
\ma{A}\cdot\vec{u}=\vec{0}\Rightarrow\vec{u}\in ;\quad\vec{u}=\begin{bmatrix}1\\1\\1\\1\end{bmatrix}
\end{align*}

\subsection{Laplace Matrix}
\begin{align*}
\ma{A}^T\cdot\ma{A} = \begin{bmatrix}
1 & 0 & -1 & 0 & 0\\ 
0 & 1 & 0 & 0 & -1 \\ 
0 & 0 & 1 & 1 & 1 \\ 
-1 & -1 & 0 & -1 & 0
\end{bmatrix}\cdot\begin{bmatrix}
1 & 0 & 0 & -1 \\ 
0 & 1 & 0 & -1 \\ 
-1 & 0 & 1 & 0 \\ 
0 & 0 & 1 & -1 \\ 
0 & -1 & 1 & 0
\end{bmatrix} = \begin{bmatrix}
2 & 0 & -1 & -1 \\ 
0 & 2 & -1 & -1 \\ 
-1 & -1 & 3 & -1 \\ 
-1 & -1 & 3 & -1
\end{bmatrix}  
\end{align*}
\begin{itemize}
\item singulär
\item $r = 3$
\item symmetrisch
\end{itemize}
\begin{align*}
\ma{A}^T\cdot\ma{A} = \underset{\text{Grad (degree)}}{\ma{D}} - \underset{\text{Adjazenz}}{\ma{W}} = \begin{bmatrix}
2 & 0 & 0 & 0 \\ 
0 & 2 & 0 & 0 \\ 
0 & 0 & 3 & 0 \\ 
0 & 0 & 0 & 3
\end{bmatrix} - \begin{bmatrix}
0 & 0 & 1 & 1 \\ 
0 & 0 & 1 & 1 \\ 
1 & 1 & 0 & 1 \\ 
1 & 1 & 1 & 0
\end{bmatrix} 
\end{align*}

\subsection{Kirchhoffsches Stromgesetz (KCL)}

\begin{tabular}{lll}
$\ma{A}^T\cdot \vec{w} = \vec{0}$ & (keine Stromquellen) & $\vec{w}$: Kantenströme\\
$\ma{A}^T\cdot \vec{w} = \vec{f}$ & (mit Stromquellen) & $\vec{f}$: Stromquellen
\end{tabular}

\subsection{Ohmsches Gesetz}
\begin{tabular}{ll}
$\vec{w} = \ma{C} \cdot \vec{e}$ & $\ma{C}$: Diagonalmatrix der Kantenleitwerte\\
 & $\vec{e}$: Kantenspannungen
\end{tabular}

\subsection{Kirchhoffsches Spannungsgesetz (KVL)}
\begin{tabular}{ll}
$\vec{e} = \vec{b} - \ma{A} \cdot \vec{u}$ & $\vec{u}$: Kantenspannungen - GESUCHT\\
 & $\vec{b}$: Spannungsquellen
\end{tabular}

\begin{align*}
\ma{A}^T \cdot \vec{w} = \vec{f}\\
\ma{A}^T \cdot \ma{C} \cdot \vec{e} = \vec{f}\\
\ma{A}^T \cdot \ma{C} \left( \vec{b} - \ma{A}\cdot\vec{u}\right) = \vec{f}\\
\underset{\text{Systemmatrix, gewichtete Laplace Matrix}}{\ma{A}^T \cdot \ma{C} \cdot \ma{A}} \cal \vec{u} = \ma{A}^T \cdot \ma{C} \cdot \vec{b} \cdot \vec{f}\\
\ma{Y}_{<n\times n>} \cdot \vec{u}_{<n>} = \vec{d}_{<n>}
\end{align*}

singulär $\Rightarrow$ nicht invertierbar\\
$\Rightarrow$ keine Lösung für Gleichungssystem\\
Abhilfe: Festlegung eines Bezugspunktes, z.B. $u_0 = 0$\\
\ldots

\subsection{Alternative Darstellung}
\begin{align*}
\ma{C}^{-1} \cdot \vec{w} + \ma{A} \cdot \vec{u} &= \vec{b}\\
\ma{A}^T \cdot \vec{w} = \vec{f}\\
\begin{bmatrix}
\ma{C}^{-1} & \ma{A} \\ \ma{A}^T & \ma{0}\end{bmatrix} \cdot \begin{bmatrix}\vec{w} \\ \vec{w}\end{bmatrix} = \begin{bmatrix}\vec{b}\\\vec{f}\end{bmatrix}
\end{align*}

\ldots

\subsection{Lösung eines linearen Gleichungssystems}
\begin{align*}
\ma{A} \cdot \vec{x} = \vec{b}, \qquad \ma{A} \in \mathbb{R}^{n\times n}; \quad \vec{b}, \vec{x} \in \mathbb{R}^n, \quad \ma{A} \neq 0\\
\text{Theoretische: } \vec{x} = \ma{A}^{-1} \cdot \vec{b}\\
\text{Alternativ: Lösung durch "`Division"'-Eliminierung} 
\end{align*}

\subsection{Methode zur Bestimmung von $A^{-1}$}
\begin{align*}
\ma{A}\cdot\ma{A}^{-1} &= \ma{I}\\
\text{Sei }\ma{X} &= \ma{A}^{-1}\\
\ma{A}\cdot\ma{X} &= \ma{I}\\
\ma{A}\cdot\begin{bmatrix}
\vec{x}_1 & \vec{x}_2 & \ldots & \vec{x}_n
\end{bmatrix} &= \begin{bmatrix}
\vec{l}_1 & \vec{l}_2 & \ldots & \vec{l}_n
\end{bmatrix};\quad\vec{l}_1 = \begin{pmatrix}
1\\ 0\\ \vdots\\ 0\\ 0
\end{pmatrix}, \ldots, \vec{l}_n = \begin{pmatrix}
0\\ 0\\ \vdots\\ 0\\ 1
\end{pmatrix}
\end{align*}
$\ma{A}\cdot\vec{x}_1 = \vec{l}_1;\quad\ma{A}\cdot\vec{x}_2 = \vec{l}_2;\quad\ldots;\ma{A}\cdot\vec{x}_n = \vec{l}_n;$\\
Nur rechte Seite ändert sich; $\ma{A} = \ma{L}\cdot\ma{U}$ nur einmal erforderlich.\\
Aufwand $n\cdot 2\cdot\frac{1}{2}(n^2 + n) = n^3 + n^2 \rightarrow O(n^3)$ für Substitution.

Probleme:
\begin{itemize}
\item Pivot $p = 0$ \lightning\lightning 
\item Pivot $p$ "'sehr klein"'
\end{itemize}
Abhilfe: "'Pivotisierung"'- Zeilen-/Spaltentausch
\begin{itemize}
\item "'partial pivoting"': Wähle Zeile mit betragsmäßig größtem Element in Pivotspalte
\item "'complete pivoting"': Zeilen- und Spaltentausch
\item Pivotisierung auch zur Aufwandreduktion (Nullelemente erhalten)
\end{itemize}

Beispiel $p = 0$
\begin{align*}
\begin{pmatrix}
10 & -7 & 0 \\ 
-3 & \num{2.1} & 6 \\ 
5 & -1 & 5
\end{pmatrix}\cdot\vec{x} &= \begin{pmatrix}
7\\ \num{3.9}\\ 6
\end{pmatrix}& \begin{matrix}
p_{21} = \frac{5}{10} = \frac{1}{2}\\ p_{31} = \frac{-3}{10}
\end{matrix}\\
\begin{pmatrix}
10 & -7 & 0 \\ 
0 & 0 & 6 \\ 
0 & \num{2.5} & 5
\end{pmatrix}\cdot\vec{x}& = \begin{pmatrix}
7\\ 6\\ \num{2.5}
\end{pmatrix}& p_{32} = \frac{0}{\num{2.5}}\quad\text{\lightning}\\
\Longrightarrow\text{ Zeilentausch}\\
\begin{pmatrix}
10 & -7 & 0 \\
0 & \num{2.5} & 5 \\
0 & 0 & 6
\end{pmatrix}\cdot\vec{x}& = \begin{pmatrix}
7\\ \num{2.5}\\ 6
\end{pmatrix} 
\end{align*}
Zeilentausch durch Permutationsmatrix $\ma{P}$.\\
$\ma{P}\ma{A}\vec{x} = \ma{P}\vec{b}\Rightarrow$ dann $\ma{P}\ma{A}\rightarrow\ma{L}\ma{U}$\\
hier: $\ma{P} = \begin{pmatrix}
1 & 0 & 0\\ 0 & 0 & 1\\ 0 & 1 & 0
\end{pmatrix}$


\subsection{Kondition eines Gleichungssystems}
\begin{align*}
\begin{pmatrix}
1 & 1\\ 1 & \num{1.0001}
\end{pmatrix}\cdot\begin{pmatrix}
x_1\\ x_2
\end{pmatrix} &= \begin{pmatrix}
2\\ 2
\end{pmatrix}&\Rightarrow\vec{x} = \begin{pmatrix}
2\\ 0
\end{pmatrix}\\
\begin{pmatrix}
1 & 1\\ 1 & \num{1.0001}
\end{pmatrix}\cdot\begin{pmatrix}
x_1\\ x_2
\end{pmatrix} &= \begin{pmatrix}
2\\ \num{2.0001}
\end{pmatrix}&\Rightarrow\vec{x} = \begin{pmatrix}
1\\ 1
\end{pmatrix}
\end{align*}

Änderung in der 5. Stelle von $\vec{b}$ wird zu einer Änderung in der ersten Stelle der Lösung $\vec{x}$ "'verstärkt"'. System reagiert sehr sensitiv auf kleine Änderungen der Ausgangsdaten. Kein Lösungsalgorithmus kann etwas daran ändern.
\begin{align*}
\cond{\ma{A}} = \num{4.0002e4}
\end{align*}

\subsection{Einfluss der Pivotisierung auf die Ergebnisgenauigkeit}
\begin{align*}
\ma{B}\rightarrow\begin{pmatrix}
\num{0.0001} & 1 \\ 1 & 1
\end{pmatrix}\cdot\vec{x} &= \begin{pmatrix}
1\\ 2
\end{pmatrix}\\
\begin{pmatrix}
\num{0.0001} & 1 \\ 0 & -9999
\end{pmatrix}\cdot\vec{x} &= \begin{pmatrix}
1\\ -9998
\end{pmatrix}\quad\Rightarrow\begin{matrix}
x_2 = \num{0.99989998}\\ x_1 = \num{1.00010001}
\end{matrix}
\end{align*}

Annahme: 3 Stellen Genauigkeit
\begin{align*}
\begin{pmatrix}
\num{0.0001} & 1 \\ 0 & -\num{10000}
\end{pmatrix}\cdot\vec{x} = \begin{pmatrix}
1\\ -\num{10000}
\end{pmatrix}\quad\Rightarrow\begin{array}{l}
x_2 = 1\\ x_1 = 0\text{ \lightning}
\end{array}
\end{align*}

Zerlegung $\ma{B} = \ma{L}\ma{D}\ma{U}$ (ohne Genauigkeitsbeschränkung)
\begin{align*}
\begin{array}{c}
\text{("'out of}\\ \text{scale}\\ \text{with }\ma{B}\text{ "')}
\end{array}\quad = \begin{pmatrix}
1 & 0 \\ 
\num{10000} & 1
\end{pmatrix}\cdot\begin{pmatrix}
\num{0.0001} & 0 \\ 0 & -9999
\end{pmatrix}\cdot\begin{pmatrix}
1 & \num{10000} \\ 
0 & 1
\end{pmatrix}
\end{align*}

Änderung der Pivotisierungsreihenfolge
\begin{align*}
\begin{pmatrix}
1 & 1 \\ \num{0.0001} & 1
\end{pmatrix}\cdot\vec{x} &= \begin{pmatrix}
2\\ 1
\end{pmatrix}\\
\begin{pmatrix}
1 & 1 \\ 0 & \num{0.9999}
\end{pmatrix}\cdot\vec{x} &= \begin{pmatrix}
2\\ \num{0.9998}
\end{pmatrix}\quad\Rightarrow\begin{matrix}
x_2 = \num{0.99989998}\\ x_1 = \num{1.00010001}
\end{matrix}
\end{align*}

Annahme: 3 Stellen Genauigkeit
\begin{align*}
\begin{pmatrix}
1 & 1 \\ 0 & 1
\end{pmatrix}\cdot\vec{x} &= \begin{pmatrix}
2\\ 1
\end{pmatrix}\quad\Rightarrow\begin{rcases}
x_2 = 1\\ x_1 = 0
\end{rcases}\quad\begin{matrix}
\text{nur sehr geringer}\\ \text{Genauigkeitsverlust}
\end{matrix}
\end{align*}
NB: $\cond{\ma{B}} = \num{2.61838527}$

Uns interessieren alse zwei Themen:
\begin{itemize}
\item \textbf{Kondition} der Gleichungssystems
\item \textbf{Stabilität} des Lösungsverfahrens
\end{itemize}

\subsection{Maschinendarstellung von Zahlen}
(IEEE 754-2008-"'Binary Floating Point Arithmetic Standard"')\\
Darstellung reeller Zahlen: 64 bit:\\
\begin{tabular}{lll}
s & 1 bit & Vorzeichen\\
c & 11 bit & Exponent\\
f & 52 bit & Mantisse
\end{tabular}

\[ x = (-1)^s \cdot 2 ^{c-1023}\cdot (1 + f)\]

z.B.: 
\begin{align*}
10:& \underbrace{0}_{s} \quad \underbrace{100000000010}_{c:\ 1026-1023=3}\quad\underbrace{0100\ldots010}_{f:\ \num{1.25}}\\
\num{-0.8}:& \underbrace{1}_{s} \quad \underbrace{01111111110}_{c:\ 1022-1023=-1}\quad\underbrace{10011001\ldots10011010}_{f:\ \num{1.59999}}
\end{align*}

\subsection{Gleitkomma-Arithmetik: Eigenschaften und Fehlerarten}
$x$: beliebige reelle Zahl (mathematisch exakt)\\
$z_1,\,z_2$: Gleitkomma-Maschinenzahlen (endliche Stellenzahl)\\
\textbf{Gl}eitkomma - Grund\textbf{op}erationen\\
$\text{gl}(z_1\, \text{op}\, z_2)$,\quad op := $+$, $-$, $\times$, $/$
\begin{enumerate}[label=\alph*)]
\item $z_1 \, \text{op}\, z_2 = x;$ \quad $x$ nicht notwendigerweise Gleitkomma-Maschinenzahl
\item $+,\ -$: Exponentenangleich erforderlich - Mantissenstellen können verloren gehen.
\item Subtraktion nahezu gleichgroßer Zahlen:
	\begin{itemize}
	\item Ergebnis mit wesentlich kleinerer Mantisse
	\item Normalisierung (Linksverschiebung Mantisse, Exponentenangleich): Nachziehen nicht signifikanter Ziffern, Verlust signifikanter Ziffern
	\end{itemize}
\item Unterlauf/Überlauf
\end{enumerate}

\subsection{Matrix- und Vektornormen}

\[||\cdot||:\ \mathbb{R}^n \rightarrow \mathbb{R};\ \mathbb{R}^{n\times n} \rightarrow \mathbb{R}\]\\
Bedingungen für \textbf{Vektornormen}:
\begin{itemize}
\item \[||\vec{x}|| \geq 0,\ ||\vec{x}|| = 0 \text{ nur für } \vec{x} = \vec{0}\]
\item \[||c \cdot\vec{x}|| = \abs{c} \cdot ||\vec{x}||,\ c \in \mathbb{R}\]
\item \[||\vec{x} + \vec{y}|| \leq ||\vec{x}|| + ||\vec{y}||\] Dreiecksungleichung
\end{itemize}

Bedingungen für \textbf{Matrixnormen}:
\begin{itemize}
\item \[||\ma{A}|| \geq 0,\ ||\ma{A}|| = 0 \text{ nur für } \ma{A} = \ma{0}\]
\item \[|| c \cdot \ma{A}|| = \abs{c} \cdot ||\ma{A}||,\ c \in \mathbb{R}\]
\item \[||\ma{A} + \ma{B}|| \leq ||\ma{A}|| + ||\ma{B}||\]
\item \[||\ma{A} \cdot \ma{B}|| \leq ||\ma{A}|| \cdot ||\ma{B}||\] Multiplikativitätsbedingung
\item \[||\ma{A} \cdot \vec{x}|| \leq ||\ma{A}|| \cdot ||\vec{x}||\] Kompatibilitätsbedingung
\end{itemize}

\subsubsection{Vektornormen}
\begin{tabular}{r@{ = }ll}
$\norm{\vec{x}}_1$ & $\sum\limits_{i=1}^n\abs{xi}$ & Betragssummennorm, $l_1$-Norm\\
$\norm{\vec{x}}_2$ & $\sqrt{\sum\limits_{i=1}^n\abs{xi}^2}$ & Euklidnorm, $l_2$-Norm, Vektorlänge\\
$\norm{\vec{x}}_\infty$ & $\underset{i}{\max}{\abs{xi}}$ & Maximumsnorm, $l_\infty$-Norm, Tschebychefnorm\\
$\norm{\vec{x}}_p$ & $\sqrt[p]{\sum\limits_{i=1}^n\abs{xi}^p}$, $p\geq 1$ & Höldernormnorm, $l_p$-Norm
\end{tabular}

\subsubsection{Matrixnormen $(A\in\mathbb{R}^{n\times n})$}
\begin{tabular}{r@{ = }ll}
$||\ma{A}||_M$ & $n\cdot\underset{i,j}{\max}\abs{a_{ij}}$ & Gesamtnorm, Matrixnorm $(||\ma{I}|| = n)$\\
$(||\ma{A}||_\infty = )||\ma{A}||_Z$ & $\underset{i}{\max}\sum\limits_{j=1}^n\abs{a_{ij}}$ & Zeilennorm $(||\ma{I}||_Z = 1)$\\
$(||\ma{A}||_1 = )||\ma{A}||_S$ & $\underset{j}{\max}\sum\limits_{i=1}^n\abs{a_{ij}}$ & Spaltennorm $(||\ma{I}||_S = 1)$\\
$||\ma{A}||_E$ & $\sqrt{\sum\limits_{i=1}\sum\limits_{j=1}\abs{a_{ij}}^2}$ & Euklidnorm, Schurnorm, Frobeniusnorm $(||\ma{I}||_E = \sqrt{n})$\\
$(\ma{A}||_\lambda = )||\ma{A}||_\lambda$ & $\sqrt{\lambda_\text{max}(\ma{A}^T\cdot\ma{A})}$ & Spektralnorm, Hilbertnorm $(||\ma{I}||_\lambda = 1)$\\
\end{tabular}

\subsubsection{Kompatibilität zwischen Vektor und Matrixnorm}
\begin{tabular}{r@{\hspace{0.7cm}}l}
$l_1$ : & $||\ma{A}\vec{x}||_1 \leq ||\ma{A}||_M\cdot||\vec{x}||_1$\\
& $||\ma{A}\vec{x}||_1 \leq ||\ma{A}||_S\cdot||\vec{x}||_1$\\
$l_2$ : & $M,\lambda,E$\\
$l_\infty$ : & $M,Z$\\
\end{tabular}

\subsubsection{Kondition einer Matrix}
Ausgehend von einem linearen Gleichungssystem
\[\ma{A} \cdot \vec{x} = \vec{b}\]
kann man auf der rechten Seite einen Fehler $\delta \vec{b}$ hinzufügen.
\[\ma{A} ( \cdot \vec{x} + \delta \vec{x} = \vec{b} + \delta \vec{b}\]
Dieser resultiert dann in einem Fehler $\delta \vec{x}$ der Lösung in $\vec{x}$.
Der Zusammenhang zwischen diesen Größen lautet
\[\ma{A} \delta \vec{x} = \delta \vec{b}\]
und kann umgeformt werden zu
\[ \delta \vec{x} = \ma{A}^{-1} \cdot \delta \vec{b}\]
\[ ||\delta \vec{x}|| \le ||\ma{A}^{-1}|| \cdot ||\delta \vec{b}||\]
Gleichzeitig kann
\[\vec{b} = \ma{A} \vec{x}\]
umgeformt werden zu
\[||\vec{b}|| \le ||\ma{A}|| \cdot ||\vec{x}||\]
\[\frac{1}{||\vec{x}||} \le ||\ma{A}|| \cdot \frac{1}{||\vec{b}||}\]
Diese Gleichungen zusammengefasst ergeben dann
\[\frac{||\delta \vec{x}||}{||\vec{x}||} \le ||\ma{A}^{-1}|| \cdot ||\ma{A}|| \cdot \frac{||\delta \vec{b}||}{||\vec{b}||}\]
Darin stellen dann $\frac{||\delta \vec{x}||}{||\vec{x}||}$ den relativen Fehler des Ergebnisses, $||\ma{A}^{-1}|| \cdot ||\ma{A}||$ den \emph{Verstärkungsfaktor} für Fehler in $\vec{b}$ und $\frac{||\delta \vec{b}||}{||\vec{b}||}$ den relativen Fehler in $\vec{b}$ dar. Wir definieren die Kondition von A über
\[\cond(\ma{A}) = ||\ma{A}^{-1}|| \cdot ||\ma{A}\],
welche offensichtlich abhängig von der gewählten Matrixnorm ist. Eine $\cond(\ma{A}) \rightarrow \infty$ bedeutet dabei eine schlechte, $\cond(\ma{A}) \rightarrow 0$ eine gute \emph{Kondition}.

\subsubsection{Nichtlineare Gleichungen: Linearisierung}
Dies ist nötig beispielsweise in der Schaltungsanalyse inklusiver nichtlinearer Elemente (z.B. Diode, Transistor).

\begin{center}
\begin{circuitikz}
	\def\sourceX{0}
	\def\resistanceX{2}
	\def\diodeX{4}
	\def\upperY{4}
	\def\lowerY{0}
	
	\draw
		(\sourceX , \lowerY) to [I,i_=$I_0$] (\sourceX , \upperY)
		(\resistanceX , \lowerY) to [R=$R_0$,*-*,v=$u$] (\resistanceX , \upperY)
		(\sourceX , \upperY) to (\resistanceX , \upperY)
		(\sourceX , \lowerY) to (\resistanceX , \lowerY)
	;
	
	\draw
		(\diodeX , \upperY) to [D,v_>=$u_d$,i=$i_d$] (\diodeX, \lowerY)	
		(\resistanceX , \upperY) to (\diodeX , \upperY)
		(\resistanceX , \lowerY) to (\diodeX , \lowerY)
	;
\end{circuitikz}
\end{center}

Gesucht sei hierbei das stationäre Verhalten bei konstanter Erregung (\emph{DC-Arbeitspunkt}). Die Diode kann mithilfe von $i_D = I_S \cdot (e^{\frac{u_D}{U_T}} - 1)$ beschrieben werden. Daraus folgt dann
\[I_0 \cdot \frac{u_D}{R} = I_S \cdot \left(e^{\frac{u_D}{U_T}} - 1\right)\]
Gesucht ist somit die Nullstelle von
\[F(u_D) = f(u_D) - g(u_D) = 0\]
Nullstellensuchen sind meist iterative Verfahren welche aus einem Startwert $\vec{x}$ eine Lösung $\vec{x}^\star$ mit $F(\vec{x}^\star) = 0$ ermitteln. Zuerst einmal wollen wir das Problem eindimensional betrachten:
\[F(x) = 0\]
Gegeben ist dabei ein $F(x)$, welches stetig auf $[a, b]$ ist mit $F(a) \cdot F(b) < 0$ und gesucht ist $x^\star$ mit $F(x^\star) = 0$, $a \le x^\star \le b$.

Ein mögliches Verfahren ist dabei die \emph{Intervallhalbierung}.
\begin{enumerate}
\item Startintervall $[a^{(0)}, b^{(0)}] = [a, b]$, $k = 0$
\item Intervallmitte $m = \frac{a^{(k)} \cdot b^{(k)}}{2}$
\item $[a^{k + 1)}, b^{(k + 1)}] = \begin{cases} [m, b^{k)}] & \text{fuer\ } F(m) \cdot F(a^{(k)}) > 0 \\ [m, b^{k)}] & \text{fuer\ } F(m) \cdot F(a^{(k)}) < 0 \end{cases}$
\item Falls $\abs{a^{k + 1)} - b^{k + 1)}} > \epsilon$ dann gehe zu Schritt 2, $k = k + 1$
\end{enumerate}

Die Konvergenz eines iterativen Verfahrens wird mithilfe der Parameter $L$, dem Konvergenzfaktor, und $p$, der Konvergenzordnung, beschrieben.
\[\Delta^{(k + 1)} = \frac{1}{2} \Delta^{(k)} = (\frac{1}{2})^{k + 1} \Delta^{(0)}\]
\[\epsilon = x - x^\star\]
\[\abs{\epsilon^{(k + 1)}} \le L \abs{\epsilon^{(k)}}^p\]
Die Intervallhalbierung liegt bei $p = 1$, also linearer Konvergenz, und $L = 1/2$. Die Anzahl der Iterationsschritte bis zu einem Restfehler $\Delta^R$ ist damit
\[\kappa = \ceil{\ld \left( \frac{\Delta^{(0)}}{\Delta^R}\right)}\]

Ein weiteres weit verbreitetes Verfahren ist das \emph{Newton-Raphson-Verfahren}. Die Idee dahinter ist eine Taylorreihe erster Ordnung:
\[F(x) = F(x^{(k)}) + F'(x^{(k)}) \cdot (x - x^{(k)}) + F''(x^{(k)}) \cdot \frac{(x - x^{(k)})^2}{2} + \dots\]
\[x^{(k + 1)} = x^{(k)} - \frac{F(x^{(k)})}{F'(x^{(k)})}\]

Das Konvergenzverhalten des Newton-Raphson-Verfahrens erhält man aus einer Taylorreihe:
\[x^{(k + 1)} = x^{(k)} - \frac{F(x^\star) + F'(x^\star)(x^{(k)} - x^\star) + \frac{1}{2} F''(x^\star) \cdot {\epsilon^{k}}^2 + ... + \frac{1}{n!} F^{(n)}(x^\star) \cdot {\epsilon^{k}}^n}{F'(x^\star) + F''(x^\star) \cdot \epsilon^{(k)} + \frac{1}{(n - 1)!} \cdot F^{(n)} {\epsilon^{(k)}}^{n - 1}}\]
\[\epsilon^{(k + 1)} = \epsilon^{(k)} \left( 1 - \frac{F' + \frac{1}{2} F'' \cdot \epsilon^{(k)} + \dots}{F' + F'' \cdot \epsilon^{(k)} + \dots} \right) \approx \epsilon^{(k)} \left( \left(1 + \frac{1}{2} \epsilon^{(k)} \frac{F''}{F'} \right) \cdot \left( 1 - \frac{1}{2} \epsilon^{(k)} \frac{F''}{F'} \right) \right)\]
\[\epsilon^{(k + 1)} \approx \epsilon^{(k)} \left( 1 - \left( 1 + \frac{1}{2} \cdot \epsilon^{(k)} \frac{F''}{F'} - \epsilon^{(k)} \frac{F''}{F'}\right)\right) = \frac{1}{2} \cdot {\epsilon^{(k)}}^2 \frac{F''}{F'}\]

Ein Problem beim Newton-Raphson-Verfahren tritt bei n-fachen Nullstellen ($F'(x^\star) = \dots = F^{(n - 1)}(x^\star) = 0)$, $F^{(n)}(x^\star) \ne 0$) auf.
\[\epsilon^{(k + 1)} \approx \epsilon^{(k)} \cdot \left( 1 - \frac{(n - 1)!}{n!}\right) - \epsilon^{(k)} \cdot \left( 1 - \frac{1}{n}\right) = \epsilon^{(k)} \frac{n - 1}{n}\]
z.B.: $F(x) = e^x - 1 \Rightarrow x^{(k + 1)} = x^{(k)} - \frac{e^{x^{(k)}} - 1}{e^{x^{(k)}}}$
	\chapter{Nichtlineare Gleichungen}
Dies ist nötig beispielsweise in der Schaltungsanalyse inklusiver nichtlinearer Elemente (z.B. Diode, Transistor).

\begin{figure}[htdp]
	\center
	\begin{circuitikz}
	\def\sourceX{0}
	\def\resistanceX{2}
	\def\diodeX{4}
	\def\upperY{4}
	\def\lowerY{0}
	
	\draw
		(\sourceX , \lowerY) to [I,i_=$I_0$] (\sourceX , \upperY)
		(\resistanceX , \lowerY) to [R=$R_0$,*-*,v=$u$] (\resistanceX , \upperY)
		(\sourceX , \upperY) to (\resistanceX , \upperY)
		(\sourceX , \lowerY) to (\resistanceX , \lowerY)
	;
	
	\draw
		(\diodeX , \upperY) to [D,v_>=$u_d$,i=$i_d$] (\diodeX, \lowerY)	
		(\resistanceX , \upperY) to (\diodeX , \upperY)
		(\resistanceX , \lowerY) to (\diodeX , \lowerY)
	;
\end{circuitikz} 

	\caption{Beispiel einer Schaltung}
\end{figure}

Gesucht sei hierbei das stationäre Verhalten bei konstanter Erregung (\emph{DC-Arbeitspunkt}). Die Diode kann mithilfe von $i_D = I_S \cdot \left(e^{\frac{u_D}{U_T}} - 1\right)$ beschrieben werden. Daraus folgt dann
\[I_0 - \frac{u_D}{R} = I_S \cdot \left(e^{\frac{u_D}{U_T}} - 1\right)\]
Gesucht ist somit die Nullstelle von
\[F(u_D) = f(u_D) - g(u_D) = 0\]
Nullstellensuchen sind meist iterative Verfahren welche aus einem Startwert $\vec{x}$ eine Lösung $\vec{x}^\star$ mit $F(\vec{x}^\star) = 0$ ermitteln. Zuerst einmal wollen wir das Problem eindimensional betrachten:
\[F(x) = 0\]
Gegeben ist dabei ein $F(x)$, welches stetig auf $[a, b]$ ist mit $F(a) \cdot F(b) < 0$ und gesucht ist $x^\star$ mit $F(x^\star) = 0$, $a \le x^\star \le b$.

\begin{figure}[htdp]
	\center
	\begin{tikzpicture}
	\draw[thick,->] (0,0) -- (4,0) node[right]{$u_D$};
	\draw[thick,->] (0,0) -- (0,3) node[above]{$i_D$};
	
	\draw[blue,domain=0:2,name path=exp] plot (\x,{exp(0.7*\x)-1});
	\draw[red,domain=0:3,name path=line] (0,2) node [left] {$I_0$} -- (3,0) node [below] {$I_0 R_0$};
	\path [name intersections={of=exp and line,by=intersect}];
	
	\filldraw [color=red] (intersect) circle (2pt);
	\draw (intersect) node [right=0.2] {Lösungspunkt};
	\draw [dashed] (intersect) -- ($(0,0)!(intersect)!(4,0)$) node [below] {$U_D^*$};
	
	\begin{scope}[shift={(6,1.5)}]
		\draw[thick,->, name path=xaxis] (-0.1,0) -- (4,0) node[right]{$u_D$};
		\draw[thick,->] (0,-3) -- (0,3) node[above]{$F(u_D)$};
		
		\draw[domain=0:2.5,name path=exp] plot (\x,{exp(0.7*\x)-3}) node [right] {$F(u_D)$};
		
		\path [name intersections={of=exp and xaxis,by=intersect}];
		\draw [<-, thick, >=stealth] (intersect)  -- ++(0.5,0.5) node [right] {Lösungspunkt};
		\draw (0,-2)node [left] {$-I_0$};		
	\end{scope}	
\end{tikzpicture} 

	\caption{Arbeitspunktermittlung durch Nullstellensuche}
\end{figure}

\section{Intervallhalbierung}
\begin{enumerate}
	\item Startintervall $[a^{(0)}, b^{(0)}] = [a, b]$, $k = 0$
	\item Intervallmitte $m = \frac{a^{(k)} + b^{(k)}}{2}$
	\item $[a^{k + 1)}, b^{(k + 1)}] = \begin{cases} [m, b^{(k)}] & \text{fuer\ } F(m) \cdot F(a^{(k)}) > 0 \\ [a^{(k)}, m] & \text{fuer\ } F(m) \cdot F(a^{(k)}) < 0 \end{cases}$
	\item Falls $\abs{a^{(k + 1)} - b^{k + 1)}} > \epsilon$ dann gehe zu Schritt 2, $k = k + 1$
\end{enumerate}

\begin{figure}[htdp]
	\center
	\begin{tikzpicture}
	\coordinate (a) at (0,0);
	\coordinate (b) at (5,0);
	
	\draw [thick] (a) node [left] {$a$} -- (b) node[right] {$b$};
	\draw[thick] (a) -- ++(0,-1) node[left] {$f(a)$} [out=10,in=-115] to (1.5625,0) node[above] {\color{red}{$\xi$}} [out=65,in=185] to (5,3);
	\draw[thick] (b) -- ++(0,3) node[right] {$f(b)$};
	
	\draw[thick,dashed] ($(a)!.25!(b)$) -- ++(0,-.45);
	\draw[thick,dashed] ($(a)!.375!(b)$) -- ++(0,.6);
	\draw[thick,dashed] ($(a)!.5!(b)$) -- ++(0,1.6);
	
	\draw[thick] (0,-2) node [left] {$a_0$} -- (5,-2) node[right] {$b_0$};
	\draw[thick] (0,-1.9) -- (0,-2.1); 
	\draw[thick] (5,-1.9) -- (5,-2.1);
	\draw[thick] (2.5,-1.9) node[above] {\color{red}{$x_1$}} -- (2.5,-2.1); 
	\draw[thick] (0,-3) node [left] {$a_1$} -- (2.5,-3) node[right] {$b_1$};
	\draw[thick] (0,-2.9) -- (0,-3.1); 
	\draw[thick] (2.5,-2.9) -- (2.5,-3.1);
	\draw[thick] (1.25,-2.9) node[above] {\color{red}{$x_2$}} -- (1.25,-3.1);
	\draw[thick] (1.25,-4) node [left] {$a_2$} -- (2.5,-4) node[right] {$b_2$};
	\draw[thick] (1.25,-3.9) -- (1.25,-4.1); 
	\draw[thick] (2.5,-3.9) -- (2.5,-4.1);
	\draw[thick] (1.875,-3.9) node[above] {\color{red}{$x_3$}} -- (1.875,-4.1);
	\draw[thick] (1.25,-5) node [left] {$a_3$} -- (1.875,-5) node[right] {$b_3$};
	\draw[thick] (1.25,-4.9) -- (1.25,-5.1); 
	\draw[thick] (1.875,-4.9) -- (1.875,-5.1); 
\end{tikzpicture} 

	\caption{Intervallhalbierung}
\end{figure}

Die Konvergenz eines iterativen Verfahrens wird mithilfe der Parameter $L$, dem Konvergenzfaktor, und $p$, der Konvergenzordnung, beschrieben.
\[\Delta^{(k + 1)} = \frac{1}{2} \Delta^{(k)} = (\frac{1}{2})^{k + 1} \Delta^{(0)}\]
\[\epsilon = x - x^\star\]
\[\abs{\epsilon^{(k + 1)}} \le L \abs{\epsilon^{(k)}}^p\]
Die Intervallhalbierung liegt bei $p = 1$, also linearer Konvergenz, und $L = 1/2$. Die Anzahl der Iterationsschritte bis zu einem Restfehler $\Delta^R$ ist damit
\[\kappa = \ceil{\log2 \left( \frac{\Delta^{(0)}}{\Delta^R}\right)}\]

\begin{figure}[htdp]
	\center
	\begin{tikzpicture}
	\draw[thick,->] (0,0) -- (4,0) node[right]{$x$};
	\draw[thick,->] (0,0) -- (0,3) node[above]{$F(x)$};
	
	\draw (0.8, -0.2) node [below] {$x^{(k+2)}$} -- (2.8, 1) node [right] {$\bar{F}^{(k+1)}(x)$};
	\draw (1.2, -0.2) node [below] {$x^{(k+1)}$} -- (3, 2) node [right] {$\bar{F}^{(k)}(x)$};
	
	\draw [dashed] (2, 0) node [below] {$x^{(k)}$} -- ++(0,1);	
\end{tikzpicture} 

	\caption{Newton-Iteration}
\end{figure}

\section{Newton-Raphson-Verfahren} Die Idee hinter dem Newton-Raphson-Verfahren ist eine Taylorreihe erster Ordnung:
\[F(x) = F(x^{(k)}) + F'(x^{(k)}) \cdot (x - x^{(k)}) + F''(x^{(k)}) \cdot \frac{(x - x^{(k)})^2}{2} + \dots\]
\[x^{(k + 1)} = x^{(k)} - \frac{F(x^{(k)})}{F'(x^{(k)})}\]

Das Konvergenzverhalten des Newton-Raphson-Verfahrens erhält man aus einer Taylorreihe:
\[x^{(k + 1)} = x^{(k)} - \frac{F(x^\star) + F'(x^\star)(x^{(k)} - x^\star) + \frac{1}{2} F''(x^\star) \cdot {\epsilon^{k}}^2 + ... + \frac{1}{n!} F^{(n)}(x^\star) \cdot {\epsilon^{k}}^n}{F'(x^\star) + F''(x^\star) \cdot \epsilon^{(k)} + \frac{1}{(n - 1)!} \cdot F^{(n)} {\epsilon^{(k)}}^{n - 1}}\]
\[\epsilon^{(k + 1)} = \epsilon^{(k)} \left( 1 - \frac{F' + \frac{1}{2} F'' \cdot \epsilon^{(k)} + \dots}{F' + F'' \cdot \epsilon^{(k)} + \dots} \right) \approx \epsilon^{(k)} \left( \left(1 + \frac{1}{2} \epsilon^{(k)} \frac{F''}{F'} \right) \cdot \left( 1 - \frac{1}{2} \epsilon^{(k)} \frac{F''}{F'} \right) \right)\]
\[\epsilon^{(k + 1)} \approx \epsilon^{(k)} \left( 1 - \left( 1 + \frac{1}{2} \cdot \epsilon^{(k)} \frac{F''}{F'} - \epsilon^{(k)} \frac{F''}{F'}\right)\right) = \frac{1}{2} \cdot {\epsilon^{(k)}}^2 \frac{F''}{F'}\]

Probleme beim Newton-Raphson-Verfahren treten bei n-fachen Nullstellen ($F'(x^\star) = \dots = F^{(n - 1)}(x^\star) = 0)$, $F^{(n)}(x^\star) \ne 0$) auf.
\[\epsilon^{(k + 1)} \approx \epsilon^{(k)} \cdot \left( 1 - \frac{(n - 1)!}{n!}\right) - \epsilon^{(k)} \cdot \left( 1 - \frac{1}{n}\right) = \epsilon^{(k)} \frac{n - 1}{n}\]
Somit erhält man nur mehr lineare Konvergenz.
z.B.: $F(x) = e^x - 1 \Rightarrow x^{(k + 1)} = x^{(k)} - \frac{e^{x^{(k)}} - 1}{e^{x^{(k)}}}$

\subsection{Variation: Sekanten-Methode}
\[F'(x^{(k)}) = \lim_{x \rightarrow x^{(k)}} \frac{F(x) - F(x^{(k)})}{x - x^{(k)}}\]
\[F'(x^{(k)}) = \frac{F(x^{(k - 1)}) - F(x^{(k)})}{x^{(k - 1)} - x^{(k)}}\]
Eingesetzt in die Newton-Gleichung erhält man somit
\[x^{(k + 1)} = x^{(k)} - \frac{F(x^{(k)}) \cdot (x^{(k)} - x^{(k - 1)})}{F(x^{(k)}) - F(x^{(k - 1)})}\]
Dadurch ist pro Schritt nur eine Funktionsauswertung nötig.

\section{Fixpunktiteration}
Fixpunkt $x^*$ einer Funktion $g(x)$
\begin{equation}
	x^* = g(x^*)
\end{equation}
Nullstellenproblem und Fixpunktproblem können ineinander überführt werden., z.B.
\[g(x) = x - F(x)\]

\subsection{Fixpunkttheorem}
\begin{enumerate}
	\item Sei $g(x)$ stetig auf $[a,b]$, sowie $g(x)\in [a,b]$ für alle $x\in [a,b]$.\\
	Dann hat $g(x)$ mindestens einen Fixpunkt auf $[a,b]$
	\item Existiere zusätzlich $g'(x)$ auf $(a,b)$, sowie eine positive Konstante $k<1$ mit \mbox{$\abs{g'(x)}\leq k$}, für alle $x\in (a,b)$
	\begin{itemize}
		\item Dann hat $g(x)$ exakt einen Fixpunkt auf $[a,b]$
		\item Gilt $0<k<1$, dann konvergiert für jedes $x^{(0)}\in [a,b]$ die Folge 
		\[x^{(n)} = g(x^{(n-1)})\quad,n\geq 1\]
		zum eindeutigen Fixpunkt $x^*\in [0,b]$.
	\end{itemize}
\end{enumerate}

\subsection{Konvergenzverhalten iterativer Verfahren}
\begin{align}
	x^{(k+1)} &= g(x^{(k)}) & \text{Iterationsvorschrift}\\
	x^* &= g(x^*) & \text{Fixpunkt}
\end{align}

Taylorreihe von $g(x)$ um $x^*$ 
\begin{align}
	\left[g(x^{(k)})\right]x^{(k+1)} &= \underbrace{g(x^*)}_{x^*} + g'(x^*)\cdot (x^{(k)}-x^*) + \frac{1}{2} g''(x^*)\cdot (x^{(k)}-x^*)^2 + \ldots\\
	\varepsilon^{(k+1)} &= g'(x^*)\cdot \varepsilon^{(k)} + \frac{1}{2} g''(x^*)\cdot {\varepsilon^{(k)}}^2 + \ldots
\end{align}
\begin{itemize}
	\item $g'(x^*)\neq 0$, $\abs{g'(x^*)} < 1 \quad\rightarrow$ lineare Konvergenz
	\item $g'(x^*) = 0$, $g''(x^*)\neq 0 \quad\Rightarrow$ quadratische Konvergenz ("Konstruktionshinweis")
\end{itemize}

\begin{figure}[htdp]
	\center
	\begin{tikzpicture}
	\draw[thick,->] (0,0) -- (4,0) node[right]{$x^{(i)}$};
	\draw[thick,->] (0,0) -- (0,4) node[above]{$x^{(i + 1)} = g(x^{(i)})$};
	
	\draw (0, -0) node [below] {$x^{(k+2)}$} -- (4, 4);
	\draw[blue,domain=0:4,name path=exp] plot (\x,{exp(0.3*\x)});
	
	\def\xZero{3.5}
	\def\xOne{2.85765111806316378986}
	\def\xTwo{2.35677777239982118523}
	\def\xThree{2.02796602315439250323}
	\def\xFour{1.83747033351067410687}
			
	\draw [dashed] (\xZero, 0) node [below] {$x^{(0)}$} -- (\xZero, \xOne);
	\draw [dashed] (0, \xOne) node [left] {$g(x^{(0)})$} -- (\xZero, \xOne);
	
	\draw [dashed] (\xOne, 0) node [below] {$x^{(1)}$} -- (\xOne, \xTwo);
	\draw [dashed] (0, \xTwo) node [left] {$g(x^{(1)})$} -- (\xOne, \xTwo);
	
	\draw [dashed] (\xTwo, 0) node [below] {$x^{(2)}$} -- (\xTwo, \xThree);
	\draw [dashed] (0, \xThree) node [left] {$g(x^{(2)})$} -- (\xTwo, \xThree);
	
	\draw [dashed] (\xThree, 0) node [below] {$x^{(3)}$} -- (\xThree, \xFour);
	\draw [dashed] (0, \xFour) node [left] {$g(x^{(3)})$} -- (\xThree, \xFour);
\end{tikzpicture} 

	\caption{Fixpunktiteration}
\end{figure}

\section{Mehrdimensionales Newton-Raphson-Verfahren}
\begin{align}
	\vec{F}(\vec{x}) &= \vec{0}\quad,\vec{F} = (F_1,\ldots,F_r)^T\\
	\vec{F}(\vec{x}^{(k+1)}) &= \vec{F}(\vec{x}^{(k)}) + \underbrace{\frac{\partial\vec{F}(\vec{x}^{(k)})}{\partial\vec{x}^T}} \cdot\left(\vec{x}^{(k+1)}-\vec{x}^{(k)}\right)\\
	\begin{matrix}
	\text{Jacobi-Matrix}\\ \text{Fundamentalmatrix}
	\end{matrix}\qquad & \ma{J}(\vec{x}) = \frac{\partial\vec{F}(\vec{x})}{\partial\vec{x}^T} = \left(\frac{\partial\vec{F}}{\partial\vec{x}_1},\ldots,\frac{\partial\vec{F}}{\partial\vec{x}_r}\right) =\begin{pmatrix}
	\frac{\partial\vec{F}_1}{\partial\vec{x}_1} & \ldots & \frac{\partial\vec{F}_1}{\partial\vec{x}_r}\\
	\vdots & & \vdots\\
	\frac{\partial\vec{F}_r}{\partial\vec{x}_1} & \ldots & \frac{\partial\vec{F}_r}{\partial\vec{x}_r}
	\end{pmatrix}\\
	\vec{F}(\vec{x}^{(k+1)}) &= \vec{0}\Rightarrow\ma{J}(\vec{x}^{(k)})\cdot\left(\vec{x}^{(k+1)}-\vec{x}^{(k)}\right) = -\vec{F}(\vec{x}^{(k)})
\end{align}
\[\vec{x}^{(k+1)} = \vec{x}^{(k)}-\ma{J}^{-1}(\vec{x}^{(k)})\cdot\vec{F}(\vec{x}^{(k)})\]
\textbf{Praktisch:}
\[\ma{J}(\vec{x}^{(k)})\cdot\vec{x}^{(k+1)} = \ma{J}(\vec{x}^{(k)})\cdot\vec{x}^{(k)} - \vec{F}(\vec{x}^{(k)})\]

\subsection{Konvergenz}
\[\vec x^{(k + 1)} = \vec \Phi(\vec x^{(k)}) = \vec \Phi(\vec x^\star) + \frac{\partial \vec \Phi}{\partial \vec x^T} \bigg|_{\vec x  = \vec x^\star} \cdot \left( \vec x^{(k)} - \vec x^\star \right) + \mathcal O\left( \left( \vec x^{(k)} - \vec x^\star \right) \cdot \left( \vec x^{(k)} - \vec x^\star \right)^T\right)\]

\[\vec \epsilon^{(k + 1)} = \frac{\partial \vec \Phi}{\partial \vec x^T} \bigg|_{\vec x = \vec x^\star} \vec \epsilon^{(k)} + \mathcal O \left( \vec \epsilon^{(k)} \cdot {\vec \epsilon^{(k)}}^T \right)\]

\[\vec \Phi(\vec x) = \vec x - \ma{J}^{-1} \cdot \vec F\]
\[\frac{\partial \vec \Phi}{\partial \vec x^T} = \underbrace{\ma{E} - \underbrace{\ma{J}^{-1} \underbrace{\frac{\partial \vec F}{\partial \vec x^T}}_{\ma J^{-1}}}_{\ma E}}_{\ma 0} - \underbrace{\frac{\partial \ma{J}^{-1}}{\partial \vec x^T} \cdot \vec F}_{\ma 0}\]

\subsection{Abbruchkriterien}
Gegeben sei eine Nullstellen-Suche: $\vec F(\vec x) = \vec 0 \Rightarrow \vec x^\star$

Als Abbruchkriterium wäre schön
\[||\vec x^{(k)} - \vec x^\star|| = || \vec \epsilon^{(k)} || \le || \vec \epsilon^{(A)} ||\]
oder
\[\frac{|| \vec \epsilon^{(k)} ||}{|| \vec x^\star ||} \le || \vec \epsilon^{(R)} ||\]
Problematisch dabei ist, dass $\vec x^\star$ unbekannt ist. Daher muss man in der Anwendung ein anderes Kriterium wählen:
\begin{itemize}
	\item $||\vec x^{(k + 1)} - \vec x^{(k)} || \le || \vec \epsilon^{(A)} ||$ bzw. $\frac{||\vec x^{(k + 1)} - \vec x^{(k)}||}{|| \vec x^{(k + 1)} ||} \le || \vec \epsilon^{(R)} ||$
	\item $||\vec F(\vec x^{(k + 1)}) - \vec F(\vec x^{(k)})|| \le \Delta_F^{(A)}$
	\item $||\vec F(\vec x^{(k)})|| \le F_{min}$
	\item $k > k_{max}$
\end{itemize} 

	\chapter{Numerische Infinitesimalrechnung}

\section{Numerisches Differenzieren}

\begin{equation}
	f'(x_0) = \lim\limits_{h\rightarrow 0} \frac{f(x_0 + h) - f(x_0)}{h}
\end{equation}

Taylor-Polynom: \[ f(x_0 + h) = f(x_0) + h\cdot f'(x_0) + \frac{h^2}{2} \cdot f''(x_0) + \frac{h^3}{3}\cdot f^{(3)}(x_0) + \ldots\]
Abbruch nach linearem Term: \[f(x_0+h) = f(x_0) + h\cdot f'(x_0) + R(h^2)\]

"'Vorwärtsdifferenz"':
\begin{equation}
	f'(x_0) = \frac{f(x_0 + h) - f(x_0)}{h} + R(h)
\end{equation}

"'Rückwärtsdifferenz:"'
\begin{equation}
	f(x_0 - h) = f(x_0) - h \cdot f(x_0) + \frac{h^2}{2} \cdot f''(x_0) - \ldots
\end{equation}

\subsection{Fehleranalyse}
Zentrierte Differenz $f'(x_0) = \frac{f(x_0+h) - f(x_0-h)}{2h} - \frac{f^{(3)}(\xi)}{6}h^2$\\
Rundungsfehler:
\begin{equation}
\begin{split}
f(x_0+h) &= \underset{\text{gerundeter Wert}}{\tilde{f}(x_0-h)} + e_{-1}\\
f(x_0-h) &= \tilde{f}(x_0+h) + e_{+1}
\end{split}
\end{equation}
\begin{equation}
f'(x_0) = \frac{\tilde{f}(x_0+h) - \tilde{f}(x_0-h)}{2h} + \frac{e_{+1} - e_{-1}}{2h} - \frac{f^{(3)}(\xi)}{6}h^2
\end{equation}
Annahmen: $\abs{e_{-1}},\abs{e_{+1}} \leq\varepsilon$ und $\abs{f^{(3)}(\xi)} \leq M$\\
Damit $\abs{f'(x_0) - \frac{\tilde{f}(x_0+h) - \tilde{f}(x_0-h)}{2h}} \leq\frac{\varepsilon}{h} + \frac{h^2M}{6}$\\
$\Rightarrow h_{\text{opt}} = \sqrt[3]{\frac{3\xi}{M}}$

\section{Numerische Integration}
Alternativer Begriff: "`Numerische Quadratur"' (bestimmte Integrale)

\subsection{Newton-Cotes-Ansätze}
\begin{equation}
\begin{split}
J &= \int\limits_a^bf(x)\diff x \cong\int\limits_a^bP_h(x)\diff x\\
P_n(x) &= a_0 + a_1x + \ldots + a_nx^n
\end{split}
\end{equation}

\begin{figure}
	\center
	\begin{subfigure}{0.3\textwidth}
		\input{figures/newton_cotes_linear}
		\caption{Polynom 1. Ordnung}
	\end{subfigure}
	\begin{subfigure}{0.3\textwidth}
		\begin{tikzpicture}	
	\draw[thick,->] (0,0) -- (4,0) node[right]{$x$};
	\draw[thick,->] (0,0) -- (0,4) node[above]{$f(x)$};
	
	\draw[domain=0.1:3.9,samples=100] plot (\x,{(-0.13)*\x*\x*\x + 0.25*\x*\x + 1*\x + 1 + 1/(\x + 0.3)});
	
	\def\a{0.7}
	\def\b{3.8}
	\def\fa{2.77791}
	\def\fb{1.5205424390243902439}
	\def\c{2.5}
	\def\fc{3.38839285714286}
	
	\draw[thick,dashed] (\a, 0) node [below] {$a$} -- (\a, \fa);
	\draw[thick,dashed] (\b, 0) node [below] {$b$} -- (\b, \fb);
	\draw[domain=\a:\b,samples=100,fill=blue,opacity=0.2,draw=none] (\a, 0) -- (\a, \fa) -- plot (\x,{-0.5729*\x*\x + 2.1724*\x + 1.5379}) -- (\b, \fb) -- (\b, 0) -- (\a, 0);
	\draw[samples=100,pattern=north east lines, pattern color=red,draw=none] (\a, \fa) -- plot[domain=\a:\c] (\x,{(-0.13)*\x*\x*\x + 0.25*\x*\x + 1*\x + 1 + 1/(\x + 0.3)}) -- (\c, \fc) -- plot[domain=\c:\a] (\x,{-0.5729*\x*\x + 2.1724*\x + 1.5379}) -- (\a, \fa);
	\draw[samples=100,pattern=north east lines, pattern color=red,draw=none] (\c, \fc) -- plot[domain=\c:\b] (\x,{(-0.13)*\x*\x*\x + 0.25*\x*\x + 1*\x + 1 + 1/(\x + 0.3)}) -- (\b, \fb) -- plot[domain=\b:\c] (\x,{-0.5729*\x*\x + 2.1724*\x + 1.5379}) -- (\c, \fc);
\end{tikzpicture} 

		\caption{Polynom 2. Ordnung}
	\end{subfigure}
	\begin{subfigure}{0.3\textwidth}
		\begin{tikzpicture}	
	\draw[thick,->] (0,0) -- (4,0) node[right]{$x$};
	\draw[thick,->] (0,0) -- (0,4) node[above]{$f(x)$};
	
	\draw[domain=0.1:3.9,name path=exp] plot (\x,{(-0.13)*\x*\x*\x + 0.25*\x*\x + 1*\x + 1 + 1/(\x + 0.3)});
	
	\def\a{0.7}
	\def\b{3.8}
	\def\fa{2.77791}
	\def\fb{1.5205424390243902439}
	\def\c{1}
	\def\fc{2.88923076923076923077}
	\def\d{2}
	\def\fd{3.39478260869565217391}
	\def\e{3}
	\def\fe{3.0430303030303030303}
	
	\draw[thick,dashed] (\a, 0) node [below] {$a$} -- (\a, \fa);
	\draw[thick,dashed] (\b, 0) node [below] {$b$} -- (\b, \fb);
	\draw[fill=blue,opacity=0.2] (\a, 0) -- (\a, \fa) -- (\c, \fc) -- (\d, \fd) -- (\e, \fe) -- (\b, \fb) -- (\b, 0) -- (\a, 0);
	\draw[thick,dashed] (\c, 0) -- (\c, \fc);
	\draw[thick,dashed] (\d, 0) -- (\d, \fd);
	\draw[thick,dashed] (\e, 0) -- (\e, \fe);
\end{tikzpicture}  

		\caption{mehrere Polynome}
	\end{subfigure}
	\caption{Unterschiedliche Methoden eine Funktion bei einer numerischen Integration anzunähern}
\end{figure}

\subsubsection{Trapezregel}
Lineares Lagrange Polynom
\begin{equation}
\begin{split}
P_1(x) &= \frac{x-x_1}{x_0-x_1}\cdot f(x_0) + \frac{x-x_0}{x_1-x_0}\cdot f(x_1)\\
\int\limits_a^b &= \int\limits_{x_0}^{x_1}P_1(x)\diff x + \underbrace{\frac{1}{2}\int\limits_{x_0}^{x_1}f''(\xi(x))\cdot (x-x_0)\cdot (x-x_1)\diff x}_{x_1 \text{Fehlerterm}}\\
 & \text{Mittelwert Integralrechnung}\\
 &= \frac{1}{2}f''(\xi)\int\limits_{x_0}^{x_1}(x-x_0)\cdot (x-x_1)\diff x = \frac{1}{2}f''(\xi)\left[\frac{x^3}{3}-\frac{(x_1+x_0)}{2}x^2+x_0x_1x\right]_{x_0}^{x_1} = \frac{-h^3}{12}f''(\xi)
\end{split}
\end{equation}
mit $h = b-a = x_1-x_0$

Also:
\begin{equation}
\int\limits_a^bf(x)\diff x = \left[\frac{(x-x_1)^2}{2(x_0-x_1)}\cdot f(x_0) + \frac{(x-x_0)^2}{2(x_1-x_0)}\cdot f(x_1)\right]_{x_0}^{x_1} - \frac{h^3}{12}f''(\xi) = \frac{h}{2}\left[f(x_0) + f(x_1)\right] - \frac{h^3}{12}f''(\xi)
\end{equation}
Exakt, sofern $f(x)$ höchstens linear ist

\subsubsection{Simpson Regel}
Quadratisches Lagrange Polynom
\begin{equation}
\begin{split}
\int\limits_a^bf(x)\diff x &= \int\limits_{x_0}^{x_2}\frac{(x-x_1)(x-x_2)}{(x_0-x_1)(x_0-x_2)}f(x_0) + \frac{(x-x_0)(x-x_2)}{(x_1-x_0)(x_1-x_2)}f(x_1) + \frac{(x-x_0)(x-x_1)}{(x_2-x_1)(x_2-x_1)}f(x_2)\diff x\\ &+ \int\limits_{x_0}^{x_2}\frac{(x-x_0)(x-x_1)(x-x_2)}{6}f^{(3)}(\xi(x))\diff x\\
&= \frac{h}{3}\left[f(x_0)+4f(x_1)+f(x_2)\right]-\frac{h^5}{90}f^{(4)}(\xi)
\end{split}
\end{equation}
mit $h=x_1-x_0 = x_2-x_1$

\paragraph{Simpson $\frac{3}{8}$ Regel}
\begin{equation}
\int\limits_a^bf(x)\diff x = \frac{3h}{8}\left[f(x_0)+3f(x_1)+3f(x_2)+f(x_3)\right]-\frac{h^5}{6480}f^{(4)}(\xi)
\end{equation}
\begin{itemize}
\item benötigt $3m$ Segmente bzw. $3m+1$ Stützpunkte
\item Grad der Genauigkeit/"`Degree of precision"'\\
Größte ganze Zahl $n$, so dass eine Integrationsformel für $x^k,\; k=0,1,\ldots,n$ exakte Resultate liefert.
\end{itemize}

\paragraph{Composite Simpson $\frac{1}{3}$}
Teile $[a,b]$ in $n$ Teilintervalle auf. Wende die Simpson Regel auf jedes Paar nacheinanderfolgender Teilintervalle an.\\
$h = \frac{b-a}{n}$; $x_j=a+j\cdot h$, $j=0,1,\ldots ,n$
\begin{equation}
\begin{split}
\int\limits_a^bf(x)\diff x &= \sum_{j=1}^{\frac{n}{2}}\int\limits_{x_{2j-2}}^{x_{2j}}f(x)\diff x\\
&= \sum_{j=1}^{\frac{n}{2}}\frac{h}{3}\left[f(x_{2j-2})+4f(x_{2j-1})+f(x_{2j})\right]\frac{h^5}{90}f^{(4)}(\xi_j)\\
&= \frac{h}{3}\left[f(x_0)+2\sum_{j=1}^{\frac{n}{2}-1}f(x_{2j}) + 4\sum_{j=1}^{\frac{n}{2}}f(x_{2j-1} + f(x_n)\right] - \frac{h^5}{90}\sum_{j=1}^{\frac{n}{2}}f^{(4)}(\xi_0)
\end{split}
\end{equation}
mit $x_{2j-2} < \xi_j y x_{2j}, f$ auf $[a,b]$ 4-mal stetig differenzierbar.

\paragraph{Fehler} $E(f) = \frac{-h^5}{90}\sum_{j=1}^{\frac{n}{2}}f^{(4)}(\xi_j)$, mit $x_{2j-2} < \xi_j y x_{2j}, j=1,2,\ldots,\frac{n}{2}$

\paragraph{Extremwertsatz}: $f^{(4)}$ nimmt Max und Min auf $[a,b]$ an, also
\begin{equation}
\begin{split}
\min\limits_{x\in[a,b]}f^{(4)}(x) \leq f^{(4)}(\xi_j) \leq \max\limits_{x\in[a,b]}f^{(4)}(x)\\
\frac{n}{2}\min\limits_{x\in[a,b]}f^{(4)}(x) \leq \sum_{j=1}^{\frac{n}{2}}f^{(4)}(\xi_j) \leq \frac{n}{2}\max\limits_{x\in[a,b]}f^{(4)}(x)\\
\min\limits_{x\in[a,b]}f^{(4)}(x) \leq \frac{n}{2}\sum_{j=1}^{\frac{n}{2}}f^{(4)}(\xi_j) \leq \max\limits_{x\in[a,b]}f^{(4)}(x)
\end{split}
\end{equation}

\paragraph{Zwischenwertsatz} Es existiert ein $\mu\in(a,b)$, so dass
\begin{equation}
f^{(4)}(\mu) = \frac{2}{n}\sum_{j=1}^{\frac{n}{2}}f^{(4)}(\xi_j)
\end{equation}
Damit
\begin{equation}
E(f) = \frac{-h^5}{90}\sum_{j=1}^{\frac{n}{2}}f^{(4)}(\xi_j) = \frac{-h^5}{180}n\cdot f^{(4)}(\mu) \overset{h=\frac{b-a}{n}}{=} \frac{-(b-a)}{180}h^4\cdot f^{(4)}(\mu)
\end{equation}

Also, mit $\mu \in (a, b)$:
\begin{equation}
	\int_a^b f(x) \diff x = \frac{h}{3} \left[ f(a) + 2 \sum_{j = 1}^{n/2 - 1} f(x_{2j} + 4 \sum_{j = 1}^{n/2} f(x_{2j - 1}) + f(b) \right] - \frac{b - a}{180} h^4 f^{(4)}(\mu)
\end{equation}

Wie man sich nun vielleicht vorstellen kann gibt es unendlich viele mögliche numerische Integrationsverfahren. Das Composite Simpson-Verfahren ist allerdings ein häufig eingesetztes allgemeines Verfahren.

\subsection{Genauigkeit}
Zur Bestimmung der Genauigkeit gibt es zwei Ansätze:
\begin{enumerate}
	\item Gegeben sei ein Verfahren, eine Funktion $f$, ein Interval $[a, b]$ und eine Stützstellenanzahl $n$ und gesucht sei der Fehler
	\item Gegeben sei eine Funktion $f$, ein Interval $[a, b]$ und ein gewünschter Fehler und gesucht sei ein Verfahren und eine Stützstellenanzahl $n$
\end{enumerate}

Der Rundungsfehler wird definiert über
\begin{equation}
	f(x_i) = \underbrace{\tilde{f}(x_i)}_{\text{gerundeter Wert}} + \underbrace{e_i}_{\text{Rundungsfehler}}
\end{equation}

Damit erhält man dann einen akkumulierten Fehler $e(h)$ bei Composite-Simpson von
\begin{equation}
	e(h) = \left| \frac{h}{3} \left[ e_0 + 2 \sum_{j = 1}^{n/2 - 1} e_{2j} + 4 \sum_{j = 1}^{n/2} e_{2j - 1} + e_n \right] \right| \le \frac{h}{3} \left[ |e_0| + 2 \sum_{j = 1}^{n/2 - 1} |e_{2j}| + 4 \sum_{j = 1}^{n/2} |e_{2j - 1}| + |e_n| \right]
\end{equation}
Seine alle Rundungsfehler $|e_i| \le \epsilon$, dann gilt
\begin{equation}
	e(h) \le \frac{h}{3} \left[ \epsilon + 2(\frac{n}{2} - 1) \epsilon + 4 \frac{n}{2} \epsilon + \epsilon \right] = n h \epsilon = (b - a) \epsilon \ne f(h, n)
\end{equation}
Erstaunlich ist somit, dass der Rundungsfehler nicht wächst, wenn der zu integrierende Bereich in mehr Teilintervalle unterteilt wird. 

\section{Numerische Lösung von DGLen}
Bsp: Einschwinganalyse (TRansiente Analyse)

\missingfigure{schaltung}

\subsection{Analytische Lösung mittels Laplace-Transformation}
\begin{equation}
\begin{split}
C\cdot\dot{y}(t) + \frac{1}{R}\cdot y(t) &= i_0(t)\\
\dot{y}(t) + \underbrace{\frac{1}{RC}}_{-p_\infty}\cdot y(t) &= \frac{1}{RC}R\cdot i_0(t)\\
\dot{y}(t) - p_\infty y(t) &= -p_\infty R\cdot i_0(t)\\
\LT \text{Laplace}\\
p\cdot Y(p) - \underbrace{y(+0)}_{=0} - p_\infty Y(p) &= -p_\infty R\frac{1}{p}\cdot I_0\\
Y(p) &= \frac{-p_\infty R\cdot I_0}{(p-p_\infty)} = \frac{A}{p} + \frac{B}{p-p_\infty}
\end{split}
\end{equation}
\begin{equation}
A = \lim\limits_{p\rightarrow 0} p\cdot Y(p) = R\cdot I_0
\end{equation}
\begin{equation}
B = \lim\limits_{p\rightarrow p_\infty} (p-p_\infty)\cdot Y(p) = -R\cdot I_0
\end{equation}
\begin{equation}
\begin{split}
\LT\\
y(t) = A + B e^{p_\infty t} = R\cdot I_0\cdot (1-e^{p_\infty t})
\end{split}
\end{equation}

\subsection{Numerische Lösung mittels explizite Euler-Methode (linear Z-Transformation)}
\missingfigure{plot}
Diskretisierung der Zeit 
\begin{equation}
y(\nu)\hat{=} y(\nu\cdot\Delta t) = y(t)
\end{equation}
Diskretisierung der DGL 
\begin{equation}
\dot{y}(\nu)\approx p_\infty\cdot y(\nu) - p_\infty - p_\infty R\cdot i_0(\nu)
\end{equation}
Differenzengleichung (explizite Euler-Methode)
\begin{equation}
y(\nu+1)\approx y(\nu) + \Delta t\cdot\dot{y}(\nu)
\end{equation}

\missingfigure{plot}
\begin{equation}
y(\nu+1)\approx y(\nu) + \Delta t\cdot\left[p_\infty y(\nu) - p_\infty R\cdots i_0(\nu)\right]\quad ,\nu = 0,1,2,\ldots
\end{equation}
\begin{equation}
\hat{y}(\nu+1) = (1+p_\infty\Delta t)\cdot y(\nu) - \Delta t p_\infty R i_0(\nu)
\label{eq:y_dach}
\end{equation}
sukzessive Berechnung ausgehend von $y(0),\;\nu = 1,2,3,\ldots$

Bei linearen Schaltungen (linearen Differenzengleichungen) ist geschlossene numerische Lösung möglich durch $\ZT$ von \autoref{eq:y_dach}
\begin{equation}
z\cdot\dot{Y}(z)-\underbrace{z\cdot y(0)}_{0} - (1-p_\infty\Delta t)\cdot Y(z) = -p_\infty R\Delta t \frac{z}{z-1}I_0
\end{equation}
\begin{equation}
\begin{split}
Y(z) &= \frac{-p_\infty\Delta t R}{z-(1+p_\infty\Delta t)}\frac{zI_0}{z-1} = \frac{zRI_0}{z-1}\frac{zRI_0}{z-(1+p_\infty\Delta t)}\\
\ZT\\
\hat{y}(\nu) &= \left[1-(1+p_\infty\Delta t)^\nu\right]\cdot RI_0\quad ,\nu = 0,1,2,\ldots
\end{split}
\end{equation}

Zahlenbeispiel: $I_0R = 1;\;p_\infty = -1,\; t=1$\\
exakte Lösung: $y(1) = 1-e^{-1} = \num{0.632121}$\\
numerische Lösung: $y(\nu) = 1-(1-\Delta t^\nu\quad ,\nu\cdot\Delta t = 1$

\begin{tabular}{c|c|c|c}
$\Delta t$ & $\nu$ & $\dot{y}(1)$ & $\varepsilon^{(A)} = \abs{\hat{y}(1)-y(1)}$\\
\hline
\num{0.1} & \num{10} & \num{0.651322} & \num{0.019201}\\
\num{0.05} & \num{20} & \num{0.641514} & \num{0.009393}\\
\num{0.025} & \num{40} & \num{0.636768} & \num{0.004647}\\
\end{tabular} 
$\quad$ Fehler $\varepsilon^{(A)}\sim \Delta t$
	 \chapter{Interpolation}
Aufgabe: Eine (stetige) Funktion zu bestimmen, welche gegebene Datenpunkte \textit{bestmöglichst} abbildet.

Eine möglicher Herangehensweise an diese Problem ist eine Entwicklung in eine Taylorreihe, welche allerdings gewisse Nachteile aufweist:
\begin{itemize}
	\item konzentriert sich auf einen Punkt
	\item keine Aussage über Genauigkeit an anderen Punkten
	\item Approximation verbessert sich nicht mit höheren Graden des Approximationspolynoms: z.B.: $e^x$, $\frac{1}{x}$
\end{itemize}
		
Eine Interpolation hingegen zielt auf eine exakte Abbildung an den gegebenen Stützstellen eine Näherung dazwischen ab.

Allgemein formuliert lautet das Interpolationsproblem: Gegeben $n + 1$ Paare von reellen oder komplexen Zahlen $\left( (x_0, f(x_0)), (x_1, f(x_1)), \ldots , (x_n, f(x_n)) \right)$. Mit Stützstellen bezeichnet man $x_i$, mit Stützwert $f(x_i)$ und mit Stützpunkt $(x_i, f(x_i))$.

Ziel ist die Bestimmung von $\Phi(x, a_0, \ldots, a_n), a_0, a_1, \ldots, a_n$, so dass $\forall_{i = 0}^n f(x_i) = \Phi(x_i)$

\section{Lagrange Interpolationspolynom}
\begin{equation}
	L_{n, k} (x) = \frac{(x - x_0) \cdot (x - x_1) \cdot \ldots \cdot (x - x_{k-1}) \cdot (x - x_{k + 1}) \cdot \ldots \cdot (x - x_n)}{(x_k - x_0) \cdot (x_k - x_1) \cdot \ldots \cdot (x_k - x_{k - 1}) \cdot (x_k - x_{k + 1}) \cdot \ldots \cdot (x_k - x_n)}
\end{equation}
\begin{equation}
	L_k(x) = L_{n, k} = \prod_{i = 0, i \ne k}^n \frac{(x - x_i)}{x_k - x_i)}
\end{equation}
$L_{n, k}(x_k) = 1$, $L_{n, k}(x_i) = 0$ für $x_i \ne x_k$
\begin{equation}
	P(x) = \sum_{k = 0}^n f(x_k) \cdot L_{n, k}(x)
\end{equation}

Theorem: Seien $n + 1$ paarweise unterschiedliche Stützstellen $x_i$ sowie die zugehörigen Stützwerte $f(x_i)$ gegeben. Dann existiert ein eindeutiges Interpolationspolynom $P(x)$ vom Grad höchstens $n$ mit $P(x_i) = f(x_i)$

Fehlerabschätzung: Gegeben seien $n + 1$ paarweise unterschiedliche Stützstellen $x_i \in [a, b]$ sowie eine Funktion $f$ die auf diesem Intervall $[a, b]$ $(n + 1)$-mal stetig differenzierbar ist. Dann existiert für $x \in [a, b]$ ein $\xi(x) \in (a, b)$ mit der Eigenschaft
\begin{equation}
	f(x) = P(x) + \frac{f^{(n + 1)}(\xi(x))}{(n + 1)!} \prod_{i = 0}^{n} (x - x_i)
\end{equation}

\section{Dividierte Differenzen} Sei $P(x)$ das Lagrange-Polynom $n$-ter Ordnung. Wähle
\begin{equation}
	P_n(x) = a_0 + a_1(x - x_0) + a_2(x - x_0)(x - x_1) + \ldots + a_n(x - x_0)(x - x_1) \cdot \ldots \cdot (x - x_{n-1})
\end{equation}
Setzt man nun kontiniuerlich die Stützstellen $x_i$ in $P_n(x)$ erhält man
\begin{equation}
	P_n(x_0) = a_0 = f(x_0)
\end{equation}
und somit den Wert für $a_0$. Aus
\begin{equation}
	P_n(x_1) = f(x_0) + a_1(x_1 - x_0) = f(x_1)
\end{equation}
folgt dann die Bestimmung von $a_1$
\begin{equation}
	a_1 = \frac{f(x_1) - f(x_0)}{x_1 - x_0}
\end{equation}

\textit{Dividierte Differenzen}-Notation:
\begin{itemize}
	\item 0. dividierte Differenz
	\begin{equation}
		f[x_i] = f(x_i)
	\end{equation}
	\item 1. dividierte Differenz
	\begin{equation}
		f[x_i, x_{i + 1}] = \frac{f[x_{i + 1}] - f[x_i]}{x_{i + 1} - x_i} = \frac{f(x_{i + 1} - f(x_i)}{x_{i + 1} - x_i}
	\end{equation}
	\item k. dividierte Differenz
	\begin{equation}
		f[x_i, x_{i + 1}, \ldots, x_{i + k}] = \frac{f[x_{i + 1}, x_{i + 2}, \ldots x_{i + k}] - f[x_i, x_{i + 1}, x_{i + 2}, \ldots x_{i + k - 1}]}{x_{i + k} - x_i}
	\end{equation}
\end{itemize}

Mithilfe dieser Notation erhält man $P_n(x)$ als sogenannte Newtons dividierte Differenz
\begin{equation}
	P_n(x) = f[x_0] + \sum_{k = 1}^n f[x_0, x_1, \ldots, x_k] \cdot (x - x_0) \cdot (x - x_{k - 1})
\end{equation}

Durch eine tabellarische Darstellung erhält man eine systematische Vorgehensweise zur Berechnung der dividierten Differenzen.

\begin{tabular}{c|c|c|c}
	$x$ & $f(x)$ & 1. div. Diff. & 2. div. Diff. \\ \hline
	$x_0$ & $f[x_0]$ & $f[x_0, x_1] = \frac{f[x_1] - f[x_0]}{x_1 - x_0}$ & $f[x_0, x_1, x_2] = \frac{f[x_1, x_2] - f[x_1, x_0]}{x_2 - x_0}$ \\
	$x_1$ & $f[x_1]$ & $f[x_1, x_2] = \frac{f[x_2] - f[x_1]}{x_2 - x_1}$ & \vdots \\
	$x_2$ & $f[x_2]$ & $f[x_2, x_3] = \frac{f[x_3] - f[x_2]}{x_3 - x_2}$ & \\
\vdots & \vdots & \vdots & 
\end{tabular}

Bei äquidistanten Stützstellen bietet sich eine vereinfachte Notation an
\begin{equation}
	h = x_{i + 1} - x_i
\end{equation}
\begin{equation}
	x = x_0 + s h
\end{equation}
\begin{equation}
	x - x_i = (s - i) \cdot h
\end{equation}
Damit ist dann
\begin{equation}
	P_n(x) = P_n(x_0 + s \cdot h) = s \cdot h \cdot f[x_0, x_1] + s \cdot (s - 1) \cdot h^2 f[x_0, x_1, x_2] + \ldots + (s - n + 1) h^n f[x_0, x_1, \ldots, x_n]
\end{equation}

Mithilfe von Binomialkoeffizienten
\begin{equation}
	\begin{pmatrix} s \\ k \end{pmatrix} = \frac{s (s - 1) \cdot \ldots \cdot (s - k + 1)}{k!}
\end{equation}
\begin{equation}
	P_n(x) = f[x_0] + \sum_{k = 1}^n \begin{pmatrix} s \\ k	\end{pmatrix} k! h^k f[x_0, x_1, \ldots, x_k]
\end{equation}

$\Delta$-Notation: $\Delta f(x_i) = f(x_{i + 1}) - f(x_i)$
\begin{equation}
	f[x_0, x_1] = \frac{f(x_1) - f(x_0)}{x_1 - x_0} = \frac{\Delta f(x_i)}{h}
\end{equation}
\begin{equation}
	f[x_0, x_1, x_2] = \frac{1}{2h} \frac{\Delta f(x_1) - \Delta f(x_0)}{h} = \frac{1}{2} \Delta^2 f(x_0)
\end{equation}
\begin{equation}
	f[x_0, x_1, \ldots, x_k] = \frac{1}{k! h!} \Delta^k f(x_0)
\end{equation}
Damit erhält man dann die Newton'sche Vorwärtsdifferenz
\begin{equation}
	P_n(x) = f(x_0) + \sum_{k = 1}^n \begin{pmatrix} s \\ k \end{pmatrix} \Delta^k f(x_0)
\end{equation}

\section{Spline - Interpolation}
\textbf{Grundidee:} Interpolation durch stückweise stetige Polynome.\\
Einfachste Variante: stückweise lineare Interpolation (siehe \autoref{fig:lineare-interpol})\\
Nachteil: keine Differenzierbarkeit an Endpunkten der einzelnen linearen Polynome gewährleistet.\\
\begin{figure}[htbp]
	\begin{center}
		\begin{tikzpicture}

\draw[thick,->] (0,0) -- (9,0);
\draw (1,0.1) -- (1,-0.1) node [below] {$x_0$};
\draw (2,0.1) -- (2,-0.1) node [below] {$x_1$};
\draw (3,0.1) -- (3,-0.1) node [below] {$x_2$};
\draw (4,0.1) -- (4,-0.1) node [below] {$x_j$};
\draw (5,0.1) -- (5,-0.1) node [below] {$x_{j+1}$};
\draw (6,0.1) -- (6,-0.1) node [below] {$\ldots$};
\draw (7,0.1) -- (7,-0.1) node [below] {$x_{j+n-1}$};
\draw (8,0.1) -- (8,-0.1) node [below] {$x_{j+n}$};
\draw[thick,->] (0,0) -- (0,4);

\draw (1,1) -- (2,2.5) -- (3,2.7) -- (4,2.8) -- (5,2.3) -- (7,2) -- (8,1);

\end{tikzpicture} 

	\end{center}
	\caption{Beispielhafte Darstellung einer stückweise linearen Interpolation}
	\label{fig:lineare-interpol}
\end{figure}

Alternative: stückweise quadratische Polynome
\begin{itemize}
	\item je 3 Koeffizienten\\
	$\Rightarrow$ nur 2 für Bestimmung der Endpunkte erforderlich\\
	$\Rightarrow$ 3. Koeffizient kann stetige Differenzierbarkeit auf $[x_0,x_n]$ sicherstellen\\
	\item ABER: Anforderung an Ableitung in $[x_0,x_n]$ können nicht berücksichtigt werden.
\end{itemize}
Weitere Alternative: stückweise kubische Polynome ("'cubic splines"')
\begin{itemize}
	\item je 4 Koeffizienten\\
	$\Rightarrow$ stetige Differenzierbarkeit auf $[x_0,x_n]$ \\
	$\Rightarrow$ stetige 2. Ableitung
\end{itemize}

\textbf{Definition:} Gegeben sei eine auf $[a,b]$ definierte Funktion $f$ sowie Stützstellen $a=x_0<x_1<\ldots<x_n = b$. Ein kubischer Spline-Interpolant $S$ erfüllt folgende Bedingungen:
\begin{enumerate}
	\item $S(x)$ ist ein kubisches Polynom, auf $[x_j,x_{j+1}]$ als $S_j(x)$ bezeichnet, für $j=0,1,\ldots,n-1$
	\item $S_j(x_j) = f(x_j)$, $S_j(x_{j+1}) = f(x_{j+1})$, für $j=0,1,\ldots,n-1$
	\item $S_j(x_{j+1}) = S_{j+1}(x_{j+1})$, für $j=0,1,\ldots,n-2$
	\item $S_{j+1}'(x_{j+1}) = S_j'(x_{j+1})$, für $j=0,1,\ldots,n-2$
	\item $S_{j+1}''(x_{j+1}) = S_j''(x_{j+1})$, für $j=0,1,\ldots,n-2$
	\item Eine der folgenden Randbedingungen ist erfüllt
	\begin{itemize}
		\item $S''(x_0) = S''(x_n) = 0$ (freier bzw. natürlicher Rand)
		\item $S'(x_0) = f'(x_0)$, $S'(x_n) = f'(x_n)$ (eingespannter Rand)
	\end{itemize}
\end{enumerate}

\begin{equation}
	S_j(x) = a_j + b_j\, (x-x_j) + c_j\, (x-x_j)^2 + d_j\, (x-x_j)^3\ , \ j=0,1,\ldots,n-1
\end{equation}
\begin{equation}
	S_j(x_j) = a_j = f(x_j)\ , \ j=0,1,\ldots,n-1
\end{equation}
\begin{equation} % Split this!
	a_{j+1} = S_{j+1}(x_{j+1}) = S_j(x_{j+1}) = a_j + b_j\, (x_{j+1}-x_j) + c_j\, (x_{j+1}-x_j)^2 + d_j\, (x_{j+1}-x_j)^3\ , \ j=0,1,\ldots,n-1
\end{equation}
\begin{equation}
	\text{mit } h_j = x_{j+1} - x_j:\ a_{j+1} = a_j + b_j \cdot h_j + c_j \cdot h_j^2 + d_j \cdot d_j \cdot h_j^3\ , \ j=0,1,\ldots,n-1
	\label{eq:aj-def}
\end{equation}
\begin{equation}
	a_n := f(x_n) \text{ (Hilfsdefinition)}
\end{equation}
\begin{equation}
	b_n := S'(x_n)
\end{equation}
\begin{equation}
	S_j'(x) = b_j + 2\, c_j \cdot (x-x_j) + 3\, d_j \cdot (x-x_j)^2\ , \ j=0,1,\ldots,n-1
\end{equation}
\begin{equation}
	S_j'(x_j) = b_j
\end{equation}
\begin{equation}
	b_{j+1} = b_j + 2\, c_j \cdot h_j + 3\, d_j \cdot h_j^2
	\label{eq:bj-def}
\end{equation}
\begin{equation}
	c := \frac{S''(x_n)}{2}
\end{equation}
\begin{equation}
	S_j''(x) = 2\, c_j + 6\, d_j \cdot (x-x_j)
\end{equation}
\begin{equation}
	S_j''(x) = 2\, c_j
\end{equation}
\begin{equation}
	S_{j+1}''(x_{j+1} = 2\, c_{j+1})
\end{equation}
\begin{equation}
	S_{j+1}''(x_{j+1} = S_j''(x_{j+1})
\end{equation}
\begin{equation}
	c_{j+1} = c_j + 3\, d_j \cdot h_j
\end{equation}
\begin{equation}
	d_j = \frac{1}{3\, h_j} \cdot (c_{j+1} - c_j)\ , \ j=0,1,\ldots,n-1
	\label{eq:dj-def}
\end{equation}
\autoref{eq:aj-def} in \autoref{eq:bj-def}:
\begin{equation}
	a_{j+1} = a_j + b_j \cdot h_j + \frac{h_j^2}{3} \cdot (2\, c_j + c_{j+1})
	\label{eq:aj+1}
\end{equation}
\begin{equation}
	b_{j+1} = b_j + h_j \cdot (c_j + c_{j+1})
\end{equation}
\begin{equation}
	b_j = b_{j-1} + h_j \cdot (c_{j+1} + c_j)
	\label{eq:bj2-def}
\end{equation}

Aus \autoref{eq:aj+1}:
\begin{equation}
	b_j = \frac{1}{h_j} \cdot (a_{j+1} - a_j) - \frac{h_j}{3} \cdot (2\, c_j + c_{j+1})
	\label{eq:bj3}
\end{equation}
\begin{equation}
	b_{j-1} = \frac{1}{h_{j-1}} \cdot (a_j - a_{j-1}) - \frac{h_{j-1}}{3} \cdot (2\, c_{j-1} + c_j)
\end{equation}

In \autoref{eq:bj3}:
\begin{equation}
	h_{j-1}\, c_{j-1} +2 \cdot (h_{j-1} + h_j) \cdot c_j + h_j\, c_{j+1} = \frac{3}{h_j} \cdot (a_{j+1} - a_j) - \frac{3}{h_{j-1}} \cdot (a_j - a_{j-1})\ , \ j=0,1,\ldots,n-1
	\label{eq:hj-big}
\end{equation}

\textbf{Natürliche Splines: } $S''(x_0) = S''(x_n) = 0$
\begin{equation}
	\Rightarrow c_n = \frac{S'(x_n)}{2} = 0
\end{equation}
\begin{equation}
	0 = S''(x_0) = 2\, c_0 + 6\, d_0 \cdot (x_0 - x_0) \Rightarrow c_0 = 0
\end{equation}

Mit \autoref{eq:hj-big}:
\begin{equation}
	\begin{bmatrix}
		1 & 0 & 0 & 0 & \ldots & & \\
		h_0 & 2\cdot(h_0 + h_1) & h_1 & 0 & & &\\
		0 & h_1 & 2\cdot(h_1 + h_2) & h_2 & 0 & &\\
		0 & \ldots & & & & &\\
		0 & \ldots & & & 0 & 0 & 1
	\end{bmatrix} \cdot 
	\begin{bmatrix}
		c_0 \\ 
		c_1 \\ 
		\vdots \\  
		\\ 
		c_n
	\end{bmatrix} = 
	\begin{bmatrix}
	0 \\ \frac{3}{h_j} \cdot (a_2 - a_1) - \frac{3}{h_0} \cdot (a_1 - a_0) \\ \vdots \\ \vdots \\ 0
	\end{bmatrix}
\end{equation}

Matrix streng diagonaldominant (Diagonalelemente betragsmäßig größer als Summe aller anderen Elemente einer Zeile)\\
$\Rightarrow$ eindeutige Lösung
	\chapter{Ausgleichsrechnung}
Auch bekannt unter der Bezeichnung Approximation oder Least Square
\textbf{Ziele der Approximation:} ("`curve fitting"')
\begin{itemize}
\item Bestgeeignete Funktion eines bestimmten Typs, um gegebenen Daten anzunähern
\item Für gegebene Funktionen "`bestmögliche"', einfachere Funktionen zu finden. 
\end{itemize}

\textbf{"`Bestmögliche"' Funktion zur Annäherung gegebener Stützpunkte}
\begin{itemize}
\item Gegeben: Wertepaare $(x_i,y_i),\;i=1,\ldots ,m$
\item Annahme: Gerade beschriebt Wertepaare $y = f(x) = ax+b$ 
\end{itemize}

\textbf{Ansätze:} 
\begin{itemize}
\item $E = \sum\limits_{i=1}^m l_i = \sum\limits_{i=1}^m\left[y_i-(ax_i+b)\right]$
\item $E = \sum\limits_{i=1}^m \abs{l_i} = \sum\limits_{i=1}^m\abs{\left[y_i-(ax_i+b)\right]}$
\item Minmax: $\min E_\infty = \min\max\limits_{i=1,\ldots,m}\left\lbrace\abs{y_i-(ax_i+b)}\right\rbrace$
\end{itemize}

\section{Linear Least Squares}
Minimiere: $E_2(a,b) = \sum\limits_{i=1}^m\left(y_i-(ax_i+b)\right)^2$\\
Bestimmung von $a,b$: $\frac{\partial E_2}{\partial a} = 0,\;\frac{\partial E_2}{\partial b} = 0$\\
$\Rightarrow$ Normalengleichung:
\begin{itemize}
\item $a = ...$
\item $b = ...$
\end{itemize}

\section{Polynominale Least Squares}
Approximation durch $P_n(x) = a_nx^n + \ldots + a_1x + a_0$
\begin{equation}
E_2(a_0,a_1,\ldots,a_n)) = \sum\limits_{i=1}^m(y_i - P(x_i)^2) 
\end{equation}
Minimierung $\frac{\partial E_2}{\partial a_i} = 0\Rightarrow n+1$ Normalengleichungen in den $n+1$ Unbekannten $a_0,\ldots,a_n$

\section{Linear Least Squares - Sichtweise Lineare Algebra}
\begin{equation}
\ma{A}_{<m\times n}\cdot\vec{x}_{n} = \vec{b}_{n}\quad,m>n
\end{equation}
Beispiel:
\begin{itemize}
\item $m=2,n=1\quad\begin{bmatrix}
a_1 \\ a_2
\end{bmatrix}\cdot x = \begin{bmatrix}
b_1 \\ b_2
\end{bmatrix}$
\item $m=3,n=2\quad\begin{bmatrix}
a_{11} & a_{12} \\ a_{21} & a_{22} \\ a_{31} & a_{32}
\end{bmatrix}\cdot \begin{bmatrix}
x_1 \\ x_2
\end{bmatrix} = \begin{bmatrix}
b_1 \\ b_2 \\ b_3
\end{bmatrix}$
\end{itemize}

Fehler je Zeile: $e_i = b_i - \vec a_i\cdot \vec x$\\
Vektor der Fehler: $\vec e = \vec b - \ma{A}\cdot \vec x$\\
Summe der Fehlerquadrate: $E_2 = \norm{e}^2 = \norm{\vec b - \ma A\vec x}^2$\\
Gesucht $\hat{\vec x}$, so dass $E_2$ minimiert wird.
\begin{equation}
\ma A\hat{\vec x} = \vec p \quad ; \quad \vec e = \vec b - \vec p
\end{equation}
$\vec p$ ist die Projektion von $\vec b$ in den Spaltenraum von $\col(\ma A)$\\
$\vec e$ steht senkrecht auf den Spaltenraum $\col(\ma A)$
\begin{equation}
\begin{split}
E(x) &= \norm{\vec b - \ma A\vec x}^2 = \norm{\ma A\vec x - \vec b}\\
&= \vec x^T\ma A^T\ma A\vec x - (\ma A\vec x)^T\vec b - \vec b^T\ma A\vec x - \vec b^T\vec b\\
&= \vec x^T\underbrace{\ma A^T\ma A}_{\ma K}\vec x - 2\vec x^T\underbrace{\ma A^T\vec b}_{\vec f} - \vec b^T\vec b\\
&= \vec x^T\ma K\vec x - 2\vec x^T\vec f + \vec b^T\vec b\quad\text{zu minimieren}\\
&= (\vec x-\ma K^{-1}\vec f)^T\ma K(\vec x-\ma K^{-1}ßvec f) - \vec f^T\ma K^{-1}\vec f + \vec b^T\vec b
\end{split}
\end{equation}
Ausdruck kann minimal 0 werden für $\hat{\vec x} = \ma K^{-1}\vec f\Rightarrow$ dort also Minimum vom $E(\vec x)$
\begin{equation}
\begin{split}
E_\text{min} = E(\hat{\vec x}) &= (\vec b - \ma A\hat{\vec x})^T\cdot(\vec b - \ma A\hat{\vec x})\\
&= \vec b^T\vec b - \vec f^T\ma K^{-1}\vec f\\
&= \vec b^T\vec b - \vec b^T\ma A(\ma A^T\ma A)^{-1}\ma A^T\vec b
\end{split}
\end{equation}
$\vec e = \vec b - \ma A\vec x$ steht senkrecht auf $\col(\ma A)\Rightarrow\ma A^T\vec e = \vec 0$\\
\begin{equation}
\Rightarrow \ma A^T(\vec b - \ma A\hat{\vec x}) = 0 \Rightarrow \ma A^T\ma A\hat{\vec x} = \ma A^T\vec b
\end{equation}
$\underbrace{\ma A^T\ma A\hat{\vec x} = \ma A^T\vec b}_{\text{quadratisch, symmetrisch}} \text{ LGS }\Rightarrow\text{ Bestimmung von }\hat{\vec x}$

\textbf{Lösungsverfahren:}
\begin{itemize}
\item Gauß LU: OK, aber $\cond(\ma A^T\ma A) = \cond(\ma A)^2\Rightarrow$ Problematisch, wenn $\ma A$ schlecht konditioniert
\item Orthogonalzerlegung
\end{itemize}

\section{QR-Zerlegung}
\[\ma A = \ma Q\ma R \text{ mit }\begin{cases}
\ma A & \in \R^{n\times n}\\
\ma Q & \in \R^{n\times n} \text{orthogonale Matrix} (\ma Q^T = \ma Q^{-1})\\
\ma R & \in \R^{n\times n} \text{rechte obere Dreiecksmatrix}
\end{cases}\]
Damit:
\begin{equation}
\begin{split}
\ma A\vec x &= \vec b\\
\ma A^T\ma A\vec x &= \ma A^T\vec b\\
(\ma Q\ma R)^T\ma Q\ma R\vec x &= (\ma Q\ma R)^T\vec b\\
\ma R^T\underbrace{\ma Q^T\ma Q}_{\ma 0}\ma R\vec x &= \ma R^T\ma Q^T\vec b\\
\ma R^T\ma R\vec x &= \ma R^T\ma Q^T\vec b\quad\vert\cdot{\ma R^{-1}}^{-1}\\
\ma R\vec x &= \ma Q^T\vec b\Rightarrow \text{simple Rücksubstitution}
\end{split}
\end{equation}
OK, wie bestimme ich nun $\ma Q,\ma R$?

\textbf{3 wesentliche Ansätze:}
\begin{itemize}
\item Gram-Schmidt
\item Hausholder-Transformation
\item Givens-Rotation
\end{itemize}

\subsection{Givens-Rotation}
auch bekannt unter der Bezeichung Jacobi-Rotation \\
Eine Rotation ist gegeben durch
\begin{equation}
	\ma G(l, k, \Theta) = 
	\begin{bmatrix}
		1 & 0 & \hdots & \hdots & \hdots & \hdots & \hdots & 0 \\
		0 & 1 & & & & & & 0 \\
		\vdots & & c & & & -s & & \vdots \\
		\vdots & & & & & & & \vdots \\
		\vdots & & -s & & & c & & \vdots \\ 
		\vdots & & & & & & 1 & 0 \\ 
		0 & \hdots & \hdots & \hdots & \hdots & \hdots & 0 & 1 
	\end{bmatrix}
\end{equation}
wobei $c = \cos \Theta$, $s = \sin \Theta$. Komponentenweise $\ma G(l, k, \Theta) =  (g_{i,j}(l, k, \Theta))$ dargestellt ergibt das
\begin{equation}
	g_{i,j} = 
	\begin{cases} 
		\cos \Theta & \text{für } i = l, j = l \vee i = k, j = k \\
		\sin \Theta & \text{für } i = l, j = k \\
		- \sin \Theta & \text{für } i = k, j = l \\
		1 & \text{für } i = j \text{ außer } i, j = l, k \\
		0 & \text{sonst}
	\end{cases}
\end{equation}
Damit bedeutet $\ma G \cdot \vec x$ bzw. $\ma G^T \cdot \vec x$ eine Drehung des Vektors $\vec x$ um $\pm \Theta$ in der $(l, k)$-Ebene. Die Hauptanwendung hierfür ist ein iteratives Vorgehen um Nulleinträge in Matrizen oder Vektoren zu erreichen. Desweiteren ist $\ma G \ma G^T = \ma I$ und somit die Rotationsmatrix $\ma G$ orthogonal.
\begin{equation}
	\begin{pmatrix}
		c & s \\
		-s & c
	\end{pmatrix} 
	\begin{pmatrix}
		c & -s \\
		s & c
	\end{pmatrix} = 
	\begin{pmatrix}
		c^2 + s^2 & 0 \\
		0 & c^2 + s^2
	\end{pmatrix} =
	\begin{pmatrix}
		1 & 0 \\
		0 & 1
	\end{pmatrix} 
\end{equation}
Damit gilt
\begin{equation}
	\ma G_r \ma G_{r - 1} \cdot \ldots \cdot \ma G_2 \ma G_1 \ma A = \ma R
\end{equation}
\begin{equation}
	\ma A = \ma G^T \ma R = \ma Q^T \ma R
\end{equation}
Die Frage ist nun, wie $\Theta$ gewählt werden muss. Hierfür genügt die Betrachtung einer zweidimensionalen Struktur
\begin{equation}
	\begin{pmatrix}
		c & s \\
		-s & c
	\end{pmatrix} 
	\begin{pmatrix}
		x_l \\
		x_k
	\end{pmatrix} = 
	\begin{pmatrix}
		r \\
		0
	\end{pmatrix}
\end{equation}
Daraus kann gefolgert werden, dass
\begin{equation}
	s = c \frac{x_k}{x_l}
\end{equation}
und aus
\begin{equation}
	c^2 + s^2 = 1
\end{equation}
\begin{equation}
	c = \sqrt{1 - s^2} = \sqrt{\frac{x_l^2 - c^2 x_k^2}{x_l^2}}
\end{equation}
\begin{equation}
	c^2 x_l^2 = x_l^2 - c^2 - x_k^2
\end{equation}
\begin{equation}
	c = \pm \frac{x_l}{\sqrt{x_l^2 + x_k^2}}
\end{equation}
Das ganze an dem Beispiel
\begin{equation}
	\ma A = 
	\begin{pmatrix}
		3 & 1 & 0 \\
		1 & 3 & 1 \\
		0 & 1 & 3	
	\end{pmatrix}
\end{equation}
führt über
\begin{equation}
	\ma G = 
	\begin{pmatrix}
		c & s & 0 \\
		-s & c & 1 \\
		0 & 0 & 1	
	\end{pmatrix}
\end{equation}
zu
\begin{equation}
	\ma G \ma A =
	\begin{pmatrix}
		\sqrt{10} & 6/\sqrt{10} & 1/\sqrt{10} \\
		0 & 3/\sqrt{10} & 3/\sqrt{10} \\
		0 & 1 & 3	
	\end{pmatrix}
\end{equation} 

	
	% Verzeichnisse
	\cleardoublepage
	\pagenumbering{roman}
	% Abbildungs- und Tabellenverzeichnis
	%\setcounter{lofdepth}{2}
	\listoffigures
	%\listoftables
	
\end{document}