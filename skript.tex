%+------------------------------------------------------------------------------+
%| Numerische Methoden der Elektrotechnik                                       |
%| Skript geschrieben von Markus Hofbauer, Kevin Meyer und Benedikt Schmidt	    |
%+------------------------------------------------------------------------------+
\documentclass[fontsize=11pt,   % Schriftgröße
    DIV=12,          % Seitenaufteilung
    BCOR=5mm,        % Bindekorrektur
    ngerman,         % für Umlaute, Silbentrennung etc.
    paper=a4,        % Papierformat
    twoside,         % zweiseitig
    titlepage,       % es wird eine Titelseite verwendet
    parskip=half,    % Abstand zwischen Absätzen (halbe Zeile)
    listof=totoc,    % Verzeichnisse im Inhaltsverzeichnis aufführen
    ]{scrbook}

%% Packages
\usepackage[utf8]{inputenc}
\usepackage[german]{babel}
\usepackage{scientific}
\usepackage{subcaption}
\usepackage[european]{circuitikz}
\usetikzlibrary{trees,positioning,arrows,calc,intersections,patterns}
\usepackage{enumitem}
\usepackage{wasysym}
\usepackage{mathtools}
\usepackage{hyperref}
\hypersetup{
    colorlinks,
    citecolor=black,
    filecolor=black,
    linkcolor=black,
    urlcolor=black,
    %linktocpage % Seitenzahlen anstatt Text im Inhaltsverzeichnis verlinken
}
\usepackage{todonotes}

% Seitenstiel
\usepackage[automark, headsepline]{scrpage2}
\pagestyle{scrheadings}
\automark[section]{chapter}
% Trennlinie unter Kopfzeile
\setheadsepline[text]{0.4pt}
% Nummerierung der Kapitel beeinflussen
% -1 keine Überschrift wird numeriert
% 0  Kapitelüberschriften werden numeriert
% 1  Kapitel- und Abschnittsüberschriften werden numeriert
% 2  Kapitel- bis Unterabschnittsüberschriften werden numeriert
% 3  Kapitel- bis Unterunterabschnittsüberschriften werden numeriert
% 4  Kapitel- bis Paragraphsüberschriften werden numeriert
% 5  alle Überschriften werden numeriert
\setcounter{secnumdepth}{2}
\setcounter{tocdepth}{2}
% Schusterjungen und Hurenkinder vermeiden
\clubpenalty = 10000 
\widowpenalty = 10000 
\displaywidowpenalty = 10000

% From undertilde.sty
\def\undertilde#1{\mathord{\vtop{\ialign{##\crcr
$\hfil\displaystyle{#1}\hfil$\crcr\noalign{\kern1.5pt\nointerlineskip}
$\hfil\tilde{}\hfil$\crcr\noalign{\kern1.5pt}}}}}

\newcommand{\rang}[1]{\ensuremath{\text{rang}\left(#1\right)}}
\newcommand{\cond}[1]{\ensuremath{\text{cond}\left(#1\right)}}
\DeclareMathOperator{\log2}{ld}
%\renewcommand{\ma}[1]{\ensuremath{\undertilde{\boldsymbol {#1}}}}
\renewcommand{\ma}[1]{\ensuremath{\boldsymbol {#1}}}

\sisetup{per-mode=fraction, locale=DE}

\DeclareSIUnit \voltampere {VA}
\DeclareSIUnit \var {Var}
\DeclareSIUnit \newtonmeter {Nm}
\DeclareSIUnit \voltsecond {Vs}
\DeclareSIUnit \amperesecond {As}


\begin{document}
	
	\thispagestyle{plain}
\begin{titlepage}

\subject{\textsf{Mitschrift}}

\title{Numerische Methoden der Elektrotechnik}

\subtitle{TUM EI M.Sc. Wahlpflichtmodul}

\author{Markus Hofbauer \and Kevin Meyer \and Benedikt Schmidt}
\date{SS 2014\\[2cm]}

\publishers{
\begin{center}
\large\textbf{Dozent}\\Prof. Dr.-Ing. Ulf Schlichtmann\\
\end{center}}

\end{titlepage}
\maketitle
	\listoftodos
	% Inhaltsverzeichnis
	\cleardoublepage
	\pagenumbering{Roman}
	\pdfbookmark[1]{Inhaltsverzeichnis}{toc}
	\tableofcontents
	
	% Inhalt
	\cleardoublepage
	\pagenumbering{arabic}
	\chapter{Einführung}
\label{chap:einfuehrung}

\section{Zusammenfassung}

\subsection{Überblick}
\begin{center}
\begin{tikzpicture}[very thick,
		node distance=2.5cm,on grid,>=stealth',
		block/.style={rectangle,draw,rounded corners=1mm,minimum height=1cm,minimum width=2cm}]

\node [block] (top)						{Mathematik};
\node [block] (down3) [below=of top]	{Analysis} edge [-] (top);
\node [block] (down2) [left=of down3]	{Algebra} edge [-] (top);
\node [block] (down1) [left=of down2]	{\begin{tabular}{c}Logik und\\ Mengenlehre\end{tabular}} edge [-] (top);
\node [block] (down4) [right=of down3]	{Stochastik} edge [-] (top);
\node [block] (down5) [right=of down4]	{Numerik} edge [-] (top);
\end{tikzpicture}
\end{center}

\textbf{Definitionsversuch:} Entwicklung und mathematisches Verständnis von numerischen Algorithmen, als von Rechenmethoden zur zahlenmäßigen Lösung mathematischer Probleme.

\textbf{Zusammenspiel mit Informatik:}
\begin{center}
\begin{tikzpicture}[very thick,node distance=1.5cm,on grid,>=stealth']
\node (top)	{Numerische Mathematik};
\node (mid) [below=of top] {\begin{tabular}{l}Scientific Computing\\ Wissenschaftliches Rechnen\end{tabular}} edge [<-] (top);
\node (down) [below=of mid]	{Informatik} edge [<-] (mid);
\node [node distance=7cm, right=of mid] {\begin{tabular}{c}Logik und\\ Mengenlehre\end{tabular}} edge [<-] node [midway,above] {Computer} (mid);
\end{tikzpicture}
\end{center}

\subsection{Vorgehen}
\begin{center}
\begin{tikzpicture}[very thick,node distance=1.5cm,on grid,>=stealth']
\node (top)	{Anwendungsproblem};
\node (down1) [below=of top] {Primäres mathematisches Problem} edge [<-] node [right] {Mathematisierung} (top);
\node (down2) [below=of down1] {Sekundäre mathematische Probleme} edge [<-] node [right] {Mathematische Umformungen} (down1);
\node (down3) [below=of down2] {Algorithmus/Computerprogramm} edge [<-] node [right] {Programmierung} (down2);
\node (down4) [below=of down3] {"'Lösung"'} edge [<-] node [right] {Rechnung} (down3);
\node [below=of down4] {Lösung} edge [<-] node [right] {Beurteilung, "'Entmathematisierung"'} (down4);
\end{tikzpicture}
\end{center}

\subsection{Aspekte für die Lösung einer mathematischen Aufgabenstellung}
\begin{itemize}
\item Kondition eines Problems (Empfindlichkeit für Störungen)
\item Numerische Lösungsverfahren
\item Stabilität des Lösungsverfahrens (Empfindlichkeit für Störungen)
\item Effizienz des Lösungsverfahrens
\item Genauigkeit der Lösung
\end{itemize}

\section{Wiederholung}

\subsection{Lineares Gleichungssystem}

\begin{align*}
a_{11}\,x_1 + \ldots + a_{1n}\,x_n &= b_1\\
a_{21}\,x_1 + \ldots + a_{2n}\,x_n &= b_2\\
&\vdots\\
a_{m1}\,x_1 + \ldots + a_{mn}\,x_n &= b_m\\
\ma{A} \cdot \vec{x} &= \vec{b}\\
\ma{A} &\in \mathbb{R}^{m\times n},\ \vec{x} \in \mathbb{R}^n,\ \vec{b} \in \mathbb{R}^m
\end{align*}

\subsection{Matrizen}
\subsubsection{Elementare Matrix-Operationen}


\textbf{Addition:}
\begin{align*}
\ma{A}_{<m\times n>} + \ma{B}_{<m\times n>} &= \ma{C}_{<m\times n>}\\
c_{\mu\nu} &= a_{\mu\nu} + b_{\mu\nu}
\end{align*}

\textbf{Multiplikation mit Skalar:}
\begin{align*}
k \cdot \ma{A} = \ma{A} \cdot k = B\\
b_{\mu\nu} = k \cdot a_{\mu\nu}
\end{align*}

\textbf{Matrix-Multiplikation:}
\begin{align*}
\ma{A}_{<m\times n>} \cdot \ma{B}_{<n\times l>} &= \ma{C}_{<m\times l>}\\
c_{\mu\nu} &= \sum_{\nu=1}^{n} a_{\mu\nu} \cdot b_{\mu\lambda};\quad \mu=1\ldots m,\ \lambda=1\ldots l
\end{align*}

\textbf{Multiplikation Matrix-Vektor:}
\begin{align*}
\text{Spezialfall der Matrix-Multiplikation: } <n\times 1> \text{ bzw. } <1\times n>
\end{align*}

\textbf{Auffassen als Linearkombination der Spalten der Matrix:}
\begin{align*}
\ma{A} \cdot \vec{x} &= \vec{b}\\
\begin{bmatrix}\vec{a}_1 & \vec{a}_2 & \ldots & \vec{a}_n\end{bmatrix} \cdot \vec{x} &= \vec{b}\\
\vec{a}_1\,\vec{x} + \ldots + \vec{a}_n\,\vec{x}_n &= \vec{b}
\end{align*}

\textbf{Vektoren:}
\begin{align*}
\text{Spaltenvektor:} \quad & \vec{a}_{<n\times 1>}: \vec{a}\\
\text{Zeilenvektor:} \quad & \vec{b}_{<1\times m>}^T: \vec{b}^T
\end{align*}

\subsubsection{Vektormultiplikation}
\begin{align*}
\vec{b}^T\cdot \vec{a} &= c \quad \text{(Skalarprodukt $m=n$!)}\\
\vec{a} \cdot \vec{b}^T &= \ma{C} \quad \text{(Vektorprodukt, dyadisches Produkt)}
\end{align*}

\textbf{Matrix als Kombination von Zeilen- und Spaltenvektoren:}
\begin{align*}
\ma{A}_{<m\times n>} &= \begin{bmatrix} \vec{a}_1^T \\ \vec{a}_2^T \\ \vdots \\ \vec{a}_m^T \end{bmatrix}\\
\ma{B}_{<n\times l>} &= \begin{bmatrix} \vec{b}_1 & \vec{b}_2 & \ldots & \vec{b}_l \end{bmatrix}\\
\ma{A} \cdot \ma{B} &= \ma{C}\\
\vec{c}_{\mu\lambda} &= \vec{a}_\mu^T \cdot \vec{b}_\lambda
\end{align*}

\textbf{Rechenregeln:}
\begin{align*}
\left(\ma{A}\cdot \ma{B}\right)\cdot \ma{C} &= \ma{A} \cdot \left(\ma{B} \cdot \ma{C}\right) \quad \text{Assoziativität}\\
\ma{A} \cdot \left(\ma{B} + \ma{C}\right) &= \ma{A}\,\ma{B} + \ma{A}\,\ma{B} \quad \text{Distributivität}\\
\ma{A} \cdot \ma{B} &\neq \ma{B} \cdot \ma{A} \quad \text{Kumutativität gilt im Allgemeinen nicht}
\end{align*}

\textbf{Diagonalmatrix:}
\begin{align*}
\ma{D} &= \text{diag}\begin{bmatrix}d_1 & d_2 & \ldots & d_n\end{bmatrix} = \begin{bmatrix}d_1 & \ldots & 0 \\ 0 & \ldots & d_n\end{bmatrix}\\
\ma{D}_1 \cdot \ma{D}_2 &= \ma{D}_2 \cdot \ma{D}_1
\end{align*}

\textbf{Einheitsmatrix:}
\begin{align*}
\ma{I} = \ma{E} = \ma{1} = \text{diag}\begin{bmatrix}1 & \ldots & 1\end{bmatrix}\\
\ma{A} \cdot \ma{I} = \ma{I} \cdot \ma{A}\\
\ma{I}^n = \ma{I}\\
\ma{I}^{-1} = \ma{I}
\end{align*}

\textbf{Transponierte Matrix:}
\begin{align*}
\ma{A}_{<m\times n>}^T &= \ma{B}_{<n\times m>},\ b_{\nu\mu} = a_{\mu\nu}\\
\left(\ma{A} \cdot \ma{B}\right)^T &= \ma{B}^T \cdot \ma{A}^T\\
\ma{A} &= \ma{A}^T \quad \Rightarrow \text{ symmetrische Matrix}
\end{align*}

\textbf{Inverse Matrix:}
\begin{align*}
\ma{A} &\in \mathbb{R}^{n\times n}\\
\ma{A}^{-1} \cdot \ma{A} &= \ma{A} \cdot \ma{A}^{-1} = \ma{I}\\
\text{$\ma{A}^{-1}$ existiert nur für nicht singuläre $\ma{A}$}&\text{ und ist eindeutig.}\\
\ma{A} = \text{diag}\begin{bmatrix}d_1 & \ldots & d_n\end{bmatrix} & \Rightarrow \ma{A}^{-1} = \text{diag}\begin{bmatrix}\frac{1}{d_1} & \ldots & \frac{1}{d_n}\end{bmatrix}\\
\left(\ma{A} \cdot \ma{B}\right) &= \ma{B}^{-1} \cdot \ma{A}^{-1}
\end{align*}

\section{Schaltungsanalyse}
\begin{center}
\begin{circuitikz}
\draw
	(0,4)
	to [R=$Y_1$, i=\textcolor{green!70!blue}{$i_1$},-o] (4,4) node[above] {1}
	to [R=$Y_3$, i=\textcolor{green!70!blue}{$i_3$},v>=\textcolor{blue}{$u_3$},-o] (8,4) node[above] {3}
	to [R=$Y_5$, i<^=\textcolor{green!70!blue}{$i_5$},v=\textcolor{blue}{$u_5$},o-o] (12,4) node[above] {2} 
	to [short,i<=\textcolor{green!70!blue}{$i_2$}] (12,3)
	to [R=$Y_2$,v=\textcolor{blue}{$u_2$},*-*] (12,1)
	to (13.5,1)
	to [I,i_=$I_{02}$] (13.5,3) to (12,3)
	(12,1) to (12,0) to (8,0) node[ground] {}
	to [R=$Y_4$, i=\textcolor{green!70!blue}{$i_4$},v>=\textcolor{blue}{$u_4$},o-] (8,4) 
	(8,0) +(0.2,0) node[above] {0} to (0,0)
	to [V=$U_{01}$] (0,4)
	(4,4) 
	to [open,v=\textcolor{blue}{$u_1$}] (0,0);

\draw[color=red](8,0)
	to [open,v^>=\textcolor{red}{$u_{n1}$}] (4,4)
	(9,0)
	to [open,v>=\textcolor{red}{$u_{n3}$}] (9,4)
	(11,0)
	to [open,v^>=\textcolor{red}{$u_{n2}$}] (11,4);

\end{circuitikz}
\end{center}
\begin{tabular}{llll}
\textcolor{green!70!blue}{$\vec{i}_{<k>}$} & \textcolor{green!70!blue}{Kantenstromvektor} & \textcolor{blue}{$\vec{u}_{<k>}$} & \textcolor{blue}{Kantenspannungsvektor}\\
$\vec{i}_{0<k>}$ & Kantenquellenstromvektor & $\vec{u}_{0<k>}$ & Kantenquellenspannungsvektor\\
$\vec{i}_{n<k>}$ & Knotenquellenstromvektor & \textcolor{red}{$\vec{u}_{n<k>}$} & \textcolor{red}{Knotenquellenspannungsvektor}\\
\\
$\ma{A}_{<n\times k>}$ & Kontenmatrix, Knoteninzidenzmatrix & &\\
$\ma{Y}_{<k\times k>}$ & Kantenadmittanzmatrix & &\\
$\ma{Y}_{n<n\times n>}$ & Kontenadmittanzmatrix & &\\
\end{tabular}

\subsection{Gerichteter Graph der Schaltung}

\subsection{Inzidenz Matrix}
\begin{align*}
\ma{A}_{<m\times n>} = \begin{bmatrix}
1 & & & -1\\
 & 1 & & -1\\
-1 & & 1 & \\
 & & 1 & -1\\
 & -1 & 1 & 
\end{bmatrix}
\end{align*}
Summen der Spaltenvektoren $= \vec{0}$.\\
$\Rightarrow \ma{A}$ hat linear abhängige Spalten.

Rang der Matrix $\ma{A}$: $r = \rang{\ma{A}} = 3 = n-1$.\\
Dimension des Nullraums: Zahl der Spalten $-\ r = 1$ .\\
Vektor im Nullraum von $\ma{A}$:
\begin{align*}
\ma{A}\cdot\vec{u}=\vec{0}\Rightarrow\vec{u}\in ;\quad\vec{u}=\begin{bmatrix}1\\1\\1\\1\end{bmatrix}
\end{align*}

\subsection{Laplace Matrix}
\begin{align*}
\ma{A}^T\cdot\ma{A} = \begin{bmatrix}
1 & 0 & -1 & 0 & 0\\ 
0 & 1 & 0 & 0 & -1 \\ 
0 & 0 & 1 & 1 & 1 \\ 
-1 & -1 & 0 & -1 & 0
\end{bmatrix}\cdot\begin{bmatrix}
1 & 0 & 0 & -1 \\ 
0 & 1 & 0 & -1 \\ 
-1 & 0 & 1 & 0 \\ 
0 & 0 & 1 & -1 \\ 
0 & -1 & 1 & 0
\end{bmatrix} = \begin{bmatrix}
2 & 0 & -1 & -1 \\ 
0 & 2 & -1 & -1 \\ 
-1 & -1 & 3 & -1 \\ 
-1 & -1 & 3 & -1
\end{bmatrix}  
\end{align*}
\begin{itemize}
\item singulär
\item $r = 3$
\item symmetrisch
\end{itemize}
\begin{align*}
\ma{A}^T\cdot\ma{A} = \underset{\text{Grad (degree)}}{\ma{D}} - \underset{\text{Adjazenz}}{\ma{W}} = \begin{bmatrix}
2 & 0 & 0 & 0 \\ 
0 & 2 & 0 & 0 \\ 
0 & 0 & 3 & 0 \\ 
0 & 0 & 0 & 3
\end{bmatrix} - \begin{bmatrix}
0 & 0 & 1 & 1 \\ 
0 & 0 & 1 & 1 \\ 
1 & 1 & 0 & 1 \\ 
1 & 1 & 1 & 0
\end{bmatrix} 
\end{align*}

\subsection{Kirchhoffsches Stromgesetz (KCL)}

\begin{tabular}{lll}
$\ma{A}^T\cdot \vec{w} = \vec{0}$ & (keine Stromquellen) & $\vec{w}$: Kantenströme\\
$\ma{A}^T\cdot \vec{w} = \vec{f}$ & (mit Stromquellen) & $\vec{f}$: Stromquellen
\end{tabular}

\subsection{Ohmsches Gesetz}
\begin{tabular}{ll}
$\vec{w} = \ma{C} \cdot \vec{e}$ & $\ma{C}$: Diagonalmatrix der Kantenleitwerte\\
 & $\vec{e}$: Kantenspannungen
\end{tabular}

\subsection{Kirchhoffsches Spannungsgesetz (KVL)}
\begin{tabular}{ll}
$\vec{e} = \vec{b} - \ma{A} \cdot \vec{u}$ & $\vec{u}$: Kantenspannungen - GESUCHT\\
 & $\vec{b}$: Spannungsquellen
\end{tabular}

\begin{align*}
\ma{A}^T \cdot \vec{w} = \vec{f}\\
\ma{A}^T \cdot \ma{C} \cdot \vec{e} = \vec{f}\\
\ma{A}^T \cdot \ma{C} \left( \vec{b} - \ma{A}\cdot\vec{u}\right) = \vec{f}\\
\underset{\text{Systemmatrix, gewichtete Laplace Matrix}}{\ma{A}^T \cdot \ma{C} \cdot \ma{A}} \cal \vec{u} = \ma{A}^T \cdot \ma{C} \cdot \vec{b} \cdot \vec{f}\\
\ma{Y}_{<n\times n>} \cdot \vec{u}_{<n>} = \vec{d}_{<n>}
\end{align*}

singulär $\Rightarrow$ nicht invertierbar\\
$\Rightarrow$ keine Lösung für Gleichungssystem\\
Abhilfe: Festlegung eines Bezugspunktes, z.B. $u_0 = 0$\\
\ldots

\subsection{Alternative Darstellung}
\begin{align*}
\ma{C}^{-1} \cdot \vec{w} + \ma{A} \cdot \vec{u} &= \vec{b}\\
\ma{A}^T \cdot \vec{w} = \vec{f}\\
\begin{bmatrix}
\ma{C}^{-1} & \ma{A} \\ \ma{A}^T & \ma{0}\end{bmatrix} \cdot \begin{bmatrix}\vec{w} \\ \vec{w}\end{bmatrix} = \begin{bmatrix}\vec{b}\\\vec{f}\end{bmatrix}
\end{align*}

\ldots

\subsection{Lösung eines linearen Gleichungssystems}
\begin{align*}
\ma{A} \cdot \vec{x} = \vec{b}, \qquad \ma{A} \in \mathbb{R}^{n\times n}; \quad \vec{b}, \vec{x} \in \mathbb{R}^n, \quad \ma{A} \neq 0\\
\text{Theoretische: } \vec{x} = \ma{A}^{-1} \cdot \vec{b}\\
\text{Alternativ: Lösung durch "`Division"'-Eliminierung} 
\end{align*}

\subsection{Methode zur Bestimmung von $A^{-1}$}
\begin{align*}
\ma{A}\cdot\ma{A}^{-1} &= \ma{I}\\
\text{Sei }\ma{X} &= \ma{A}^{-1}\\
\ma{A}\cdot\ma{X} &= \ma{I}\\
\ma{A}\cdot\begin{bmatrix}
\vec{x}_1 & \vec{x}_2 & \ldots & \vec{x}_n
\end{bmatrix} &= \begin{bmatrix}
\vec{l}_1 & \vec{l}_2 & \ldots & \vec{l}_n
\end{bmatrix};\quad\vec{l}_1 = \begin{pmatrix}
1\\ 0\\ \vdots\\ 0\\ 0
\end{pmatrix}, \ldots, \vec{l}_n = \begin{pmatrix}
0\\ 0\\ \vdots\\ 0\\ 1
\end{pmatrix}
\end{align*}
$\ma{A}\cdot\vec{x}_1 = \vec{l}_1;\quad\ma{A}\cdot\vec{x}_2 = \vec{l}_2;\quad\ldots;\ma{A}\cdot\vec{x}_n = \vec{l}_n;$\\
Nur rechte Seite ändert sich; $\ma{A} = \ma{L}\cdot\ma{U}$ nur einmal erforderlich.\\
Aufwand $n\cdot 2\cdot\frac{1}{2}(n^2 + n) = n^3 + n^2 \rightarrow O(n^3)$ für Substitution.

Probleme:
\begin{itemize}
\item Pivot $p = 0$ \lightning\lightning 
\item Pivot $p$ "'sehr klein"'
\end{itemize}
Abhilfe: "'Pivotisierung"'- Zeilen-/Spaltentausch
\begin{itemize}
\item "'partial pivoting"': Wähle Zeile mit betragsmäßig größtem Element in Pivotspalte
\item "'complete pivoting"': Zeilen- und Spaltentausch
\item Pivotisierung auch zur Aufwandreduktion (Nullelemente erhalten)
\end{itemize}

Beispiel $p = 0$
\begin{align*}
\begin{pmatrix}
10 & -7 & 0 \\ 
-3 & \num{2.1} & 6 \\ 
5 & -1 & 5
\end{pmatrix}\cdot\vec{x} &= \begin{pmatrix}
7\\ \num{3.9}\\ 6
\end{pmatrix}& \begin{matrix}
p_{21} = \frac{5}{10} = \frac{1}{2}\\ p_{31} = \frac{-3}{10}
\end{matrix}\\
\begin{pmatrix}
10 & -7 & 0 \\ 
0 & 0 & 6 \\ 
0 & \num{2.5} & 5
\end{pmatrix}\cdot\vec{x}& = \begin{pmatrix}
7\\ 6\\ \num{2.5}
\end{pmatrix}& p_{32} = \frac{0}{\num{2.5}}\quad\text{\lightning}\\
\Longrightarrow\text{ Zeilentausch}\\
\begin{pmatrix}
10 & -7 & 0 \\
0 & \num{2.5} & 5 \\
0 & 0 & 6
\end{pmatrix}\cdot\vec{x}& = \begin{pmatrix}
7\\ \num{2.5}\\ 6
\end{pmatrix} 
\end{align*}
Zeilentausch durch Permutationsmatrix $\ma{P}$.\\
$\ma{P}\ma{A}\vec{x} = \ma{P}\vec{b}\Rightarrow$ dann $\ma{P}\ma{A}\rightarrow\ma{L}\ma{U}$\\
hier: $\ma{P} = \begin{pmatrix}
1 & 0 & 0\\ 0 & 0 & 1\\ 0 & 1 & 0
\end{pmatrix}$


\subsection{Kondition eines Gleichungssystems}
\begin{align*}
\begin{pmatrix}
1 & 1\\ 1 & \num{1.0001}
\end{pmatrix}\cdot\begin{pmatrix}
x_1\\ x_2
\end{pmatrix} &= \begin{pmatrix}
2\\ 2
\end{pmatrix}&\Rightarrow\vec{x} = \begin{pmatrix}
2\\ 0
\end{pmatrix}\\
\begin{pmatrix}
1 & 1\\ 1 & \num{1.0001}
\end{pmatrix}\cdot\begin{pmatrix}
x_1\\ x_2
\end{pmatrix} &= \begin{pmatrix}
2\\ \num{2.0001}
\end{pmatrix}&\Rightarrow\vec{x} = \begin{pmatrix}
1\\ 1
\end{pmatrix}
\end{align*}

Änderung in der 5. Stelle von $\vec{b}$ wird zu einer Änderung in der ersten Stelle der Lösung $\vec{x}$ "'verstärkt"'. System reagiert sehr sensitiv auf kleine Änderungen der Ausgangsdaten. Kein Lösungsalgorithmus kann etwas daran ändern.
\begin{align*}
\cond{\ma{A}} = \num{4.0002e4}
\end{align*}

\subsection{Einfluss der Pivotisierung auf die Ergebnisgenauigkeit}
\begin{align*}
\ma{B}\rightarrow\begin{pmatrix}
\num{0.0001} & 1 \\ 1 & 1
\end{pmatrix}\cdot\vec{x} &= \begin{pmatrix}
1\\ 2
\end{pmatrix}\\
\begin{pmatrix}
\num{0.0001} & 1 \\ 0 & -9999
\end{pmatrix}\cdot\vec{x} &= \begin{pmatrix}
1\\ -9998
\end{pmatrix}\quad\Rightarrow\begin{matrix}
x_2 = \num{0.99989998}\\ x_1 = \num{1.00010001}
\end{matrix}
\end{align*}

Annahme: 3 Stellen Genauigkeit
\begin{align*}
\begin{pmatrix}
\num{0.0001} & 1 \\ 0 & -\num{10000}
\end{pmatrix}\cdot\vec{x} = \begin{pmatrix}
1\\ -\num{10000}
\end{pmatrix}\quad\Rightarrow\begin{array}{l}
x_2 = 1\\ x_1 = 0\text{ \lightning}
\end{array}
\end{align*}

Zerlegung $\ma{B} = \ma{L}\ma{D}\ma{U}$ (ohne Genauigkeitsbeschränkung)
\begin{align*}
\begin{array}{c}
\text{("'out of}\\ \text{scale}\\ \text{with }\ma{B}\text{ "')}
\end{array}\quad = \begin{pmatrix}
1 & 0 \\ 
\num{10000} & 1
\end{pmatrix}\cdot\begin{pmatrix}
\num{0.0001} & 0 \\ 0 & -9999
\end{pmatrix}\cdot\begin{pmatrix}
1 & \num{10000} \\ 
0 & 1
\end{pmatrix}
\end{align*}

Änderung der Pivotisierungsreihenfolge
\begin{align*}
\begin{pmatrix}
1 & 1 \\ \num{0.0001} & 1
\end{pmatrix}\cdot\vec{x} &= \begin{pmatrix}
2\\ 1
\end{pmatrix}\\
\begin{pmatrix}
1 & 1 \\ 0 & \num{0.9999}
\end{pmatrix}\cdot\vec{x} &= \begin{pmatrix}
2\\ \num{0.9998}
\end{pmatrix}\quad\Rightarrow\begin{matrix}
x_2 = \num{0.99989998}\\ x_1 = \num{1.00010001}
\end{matrix}
\end{align*}

Annahme: 3 Stellen Genauigkeit
\begin{align*}
\begin{pmatrix}
1 & 1 \\ 0 & 1
\end{pmatrix}\cdot\vec{x} &= \begin{pmatrix}
2\\ 1
\end{pmatrix}\quad\Rightarrow\begin{rcases}
x_2 = 1\\ x_1 = 0
\end{rcases}\quad\begin{matrix}
\text{nur sehr geringer}\\ \text{Genauigkeitsverlust}
\end{matrix}
\end{align*}
NB: $\cond{\ma{B}} = \num{2.61838527}$

Uns interessieren alse zwei Themen:
\begin{itemize}
\item \textbf{Kondition} der Gleichungssystems
\item \textbf{Stabilität} des Lösungsverfahrens
\end{itemize}

\subsection{Maschinendarstellung von Zahlen}
(IEEE 754-2008-"'Binary Floating Point Arithmetic Standard"')\\
Darstellung reeller Zahlen: 64 bit:\\
\begin{tabular}{lll}
s & 1 bit & Vorzeichen\\
c & 11 bit & Exponent\\
f & 52 bit & Mantisse
\end{tabular}

\[ x = (-1)^s \cdot 2 ^{c-1023}\cdot (1 + f)\]

z.B.: 
\begin{align*}
10:& \underbrace{0}_{s} \quad \underbrace{100000000010}_{c:\ 1026-1023=3}\quad\underbrace{0100\ldots010}_{f:\ \num{1.25}}\\
\num{-0.8}:& \underbrace{1}_{s} \quad \underbrace{01111111110}_{c:\ 1022-1023=-1}\quad\underbrace{10011001\ldots10011010}_{f:\ \num{1.59999}}
\end{align*}

\subsection{Gleitkomma-Arithmetik: Eigenschaften und Fehlerarten}
$x$: beliebige reelle Zahl (mathematisch exakt)\\
$z_1,\,z_2$: Gleitkomma-Maschinenzahlen (endliche Stellenzahl)\\
\textbf{Gl}eitkomma - Grund\textbf{op}erationen\\
$\text{gl}(z_1\, \text{op}\, z_2)$,\quad op := $+$, $-$, $\times$, $/$
\begin{enumerate}[label=\alph*)]
\item $z_1 \, \text{op}\, z_2 = x;$ \quad $x$ nicht notwendigerweise Gleitkomma-Maschinenzahl
\item $+,\ -$: Exponentenangleich erforderlich - Mantissenstellen können verloren gehen.
\item Subtraktion nahezu gleichgroßer Zahlen:
	\begin{itemize}
	\item Ergebnis mit wesentlich kleinerer Mantisse
	\item Normalisierung (Linksverschiebung Mantisse, Exponentenangleich): Nachziehen nicht signifikanter Ziffern, Verlust signifikanter Ziffern
	\end{itemize}
\item Unterlauf/Überlauf
\end{enumerate}

\subsection{Matrix- und Vektornormen}

\[||\cdot||:\ \mathbb{R}^n \rightarrow \mathbb{R};\ \mathbb{R}^{n\times n} \rightarrow \mathbb{R}\]\\
Bedingungen für \textbf{Vektornormen}:
\begin{itemize}
\item \[||\vec{x}|| \geq 0,\ ||\vec{x}|| = 0 \text{ nur für } \vec{x} = \vec{0}\]
\item \[||c \cdot\vec{x}|| = \abs{c} \cdot ||\vec{x}||,\ c \in \mathbb{R}\]
\item \[||\vec{x} + \vec{y}|| \leq ||\vec{x}|| + ||\vec{y}||\] Dreiecksungleichung
\end{itemize}

Bedingungen für \textbf{Matrixnormen}:
\begin{itemize}
\item \[||\ma{A}|| \geq 0,\ ||\ma{A}|| = 0 \text{ nur für } \ma{A} = \ma{0}\]
\item \[|| c \cdot \ma{A}|| = \abs{c} \cdot ||\ma{A}||,\ c \in \mathbb{R}\]
\item \[||\ma{A} + \ma{B}|| \leq ||\ma{A}|| + ||\ma{B}||\]
\item \[||\ma{A} \cdot \ma{B}|| \leq ||\ma{A}|| \cdot ||\ma{B}||\] Multiplikativitätsbedingung
\item \[||\ma{A} \cdot \vec{x}|| \leq ||\ma{A}|| \cdot ||\vec{x}||\] Kompatibilitätsbedingung
\end{itemize}

\subsubsection{Vektornormen}
\begin{tabular}{r@{ = }ll}
$\norm{\vec{x}}_1$ & $\sum\limits_{i=1}^n\abs{xi}$ & Betragssummennorm, $l_1$-Norm\\
$\norm{\vec{x}}_2$ & $\sqrt{\sum\limits_{i=1}^n\abs{xi}^2}$ & Euklidnorm, $l_2$-Norm, Vektorlänge\\
$\norm{\vec{x}}_\infty$ & $\underset{i}{\max}{\abs{xi}}$ & Maximumsnorm, $l_\infty$-Norm, Tschebychefnorm\\
$\norm{\vec{x}}_p$ & $\sqrt[p]{\sum\limits_{i=1}^n\abs{xi}^p}$, $p\geq 1$ & Höldernormnorm, $l_p$-Norm
\end{tabular}

\subsubsection{Matrixnormen $(A\in\mathbb{R}^{n\times n})$}
\begin{tabular}{r@{ = }ll}
$||\ma{A}||_M$ & $n\cdot\underset{i,j}{\max}\abs{a_{ij}}$ & Gesamtnorm, Matrixnorm $(||\ma{I}|| = n)$\\
$(||\ma{A}||_\infty = )||\ma{A}||_Z$ & $\underset{i}{\max}\sum\limits_{j=1}^n\abs{a_{ij}}$ & Zeilennorm $(||\ma{I}||_Z = 1)$\\
$(||\ma{A}||_1 = )||\ma{A}||_S$ & $\underset{j}{\max}\sum\limits_{i=1}^n\abs{a_{ij}}$ & Spaltennorm $(||\ma{I}||_S = 1)$\\
$||\ma{A}||_E$ & $\sqrt{\sum\limits_{i=1}\sum\limits_{j=1}\abs{a_{ij}}^2}$ & Euklidnorm, Schurnorm, Frobeniusnorm $(||\ma{I}||_E = \sqrt{n})$\\
$(\ma{A}||_\lambda = )||\ma{A}||_\lambda$ & $\sqrt{\lambda_\text{max}(\ma{A}^T\cdot\ma{A})}$ & Spektralnorm, Hilbertnorm $(||\ma{I}||_\lambda = 1)$\\
\end{tabular}

\subsubsection{Kompatibilität zwischen Vektor und Matrixnorm}
\begin{tabular}{r@{\hspace{0.7cm}}l}
$l_1$ : & $||\ma{A}\vec{x}||_1 \leq ||\ma{A}||_M\cdot||\vec{x}||_1$\\
& $||\ma{A}\vec{x}||_1 \leq ||\ma{A}||_S\cdot||\vec{x}||_1$\\
$l_2$ : & $M,\lambda,E$\\
$l_\infty$ : & $M,Z$\\
\end{tabular}

\subsection{Kondition einer Matrix}
Ausgehend von einem linearen Gleichungssystem
\[\ma{A} \cdot \vec{x} = \vec{b}\]
kann man auf der rechten Seite einen Fehler $\Delta \vec{b}$ hinzufügen.
\[\ma{A} ( \cdot \vec{x} + \Delta \vec{x} = \vec{b} + \Delta \vec{b}\]
Dieser resultiert dann in einem Fehler $\Delta \vec{x}$ der Lösung in $\vec{x}$.
Der Zusammenhang zwischen diesen Größen lautet
\[\ma{A} \Delta \vec{x} = \Delta \vec{b}\]
und kann umgeformt werden zu
\[ \Delta \vec{x} = \ma{A}^{-1} \cdot \Delta \vec{b}\]
\[ ||\Delta \vec{x}|| \le ||\ma{A}^{-1}|| \cdot ||\Delta \vec{b}||\]
Gleichzeitig kann
\[\vec{b} = \ma{A} \vec{x}\]
umgeformt werden zu
\[||\vec{b}|| \le ||\ma{A}|| \cdot ||\vec{x}||\]
\[\frac{1}{||\vec{x}||} \le ||\ma{A}|| \cdot \frac{1}{||\vec{b}||}\]
Diese Gleichungen zusammengefasst ergeben dann
\[\frac{||\Delta \vec{x}||}{||\vec{x}||} \le ||\ma{A}^{-1}|| \cdot ||\ma{A}|| \cdot \frac{||\Delta \vec{b}||}{||\vec{b}||}\]
Darin stellen dann $\frac{||\Delta \vec{x}||}{||\vec{x}||}$ den relativen Fehler des Ergebnisses, $||\ma{A}^{-1}|| \cdot ||\ma{A}||$ den \emph{Verstärkungsfaktor} für Fehler in $\vec{b}$ und $\frac{||\Delta \vec{b}||}{||\vec{b}||}$ den relativen Fehler in $\vec{b}$ dar. Wir definieren die Kondition von A über
\[\cond{\ma{A}} = ||\ma{A}^{-1}|| \cdot ||\ma{A}||\]
welche offensichtlich abhängig von der gewählten Matrixnorm ist. Eine $\cond{\ma{A}} \rightarrow \infty$ bedeutet dabei eine schlechte, $\cond{\ma{A}} \rightarrow 0$ eine gute \emph{Kondition}.

\subsection{Nichtlineare Gleichungen: Linearisierung}
Dies ist nötig beispielsweise in der Schaltungsanalyse inklusiver nichtlinearer Elemente (z.B. Diode, Transistor).

\begin{center}
\begin{circuitikz}
	\def\sourceX{0}
	\def\resistanceX{2}
	\def\diodeX{4}
	\def\upperY{4}
	\def\lowerY{0}
	
	\draw
		(\sourceX , \lowerY) to [I,i_=$I_0$] (\sourceX , \upperY)
		(\resistanceX , \lowerY) to [R=$R_0$,*-*,v=$u$] (\resistanceX , \upperY)
		(\sourceX , \upperY) to (\resistanceX , \upperY)
		(\sourceX , \lowerY) to (\resistanceX , \lowerY)
	;
	
	\draw
		(\diodeX , \upperY) to [D,v_>=$u_d$,i=$i_d$] (\diodeX, \lowerY)	
		(\resistanceX , \upperY) to (\diodeX , \upperY)
		(\resistanceX , \lowerY) to (\diodeX , \lowerY)
	;
\end{circuitikz}
\end{center}

Gesucht sei hierbei das stationäre Verhalten bei konstanter Erregung (\emph{DC-Arbeitspunkt}). Die Diode kann mithilfe von $i_D = I_S \cdot \left(e^{\frac{u_D}{U_T}} - 1\right)$ beschrieben werden. Daraus folgt dann
\[I_0 - \frac{u_D}{R} = I_S \cdot \left(e^{\frac{u_D}{U_T}} - 1\right)\]
Gesucht ist somit die Nullstelle von
\[F(u_D) = f(u_D) - g(u_D) = 0\]
Nullstellensuchen sind meist iterative Verfahren welche aus einem Startwert $\vec{x}$ eine Lösung $\vec{x}^\star$ mit $F(\vec{x}^\star) = 0$ ermitteln. Zuerst einmal wollen wir das Problem eindimensional betrachten:
\[F(x) = 0\]
Gegeben ist dabei ein $F(x)$, welches stetig auf $[a, b]$ ist mit $F(a) \cdot F(b) < 0$ und gesucht ist $x^\star$ mit $F(x^\star) = 0$, $a \le x^\star \le b$.

\begin{center}
	\begin{tikzpicture}
		\draw[thick,->] (0,0) -- (4,0) node[right]{$u_D$};
		\draw[thick,->] (0,0) -- (0,3) node[above]{$i_D$};
		
		\draw[blue,domain=0:2,name path=exp] plot (\x,{exp(0.7*\x)-1});
		\draw[red,domain=0:3,name path=line] (0,2) node [left] {$I_0$} -- (3,0) node [below] {$I_0 R_0$};
		\path [name intersections={of=exp and line,by=intersect}];
		
		\filldraw [color=red] (intersect) circle (2pt);
		\draw (intersect) node [right=0.2] {Lösungspunkt};
		\draw [dashed] (intersect) -- ($(0,0)!(intersect)!(4,0)$) node [below] {$U_D^*$};
		
		\begin{scope}[shift={(6,1.5)}]
			\draw[thick,->, name path=xaxis] (-0.1,0) -- (4,0) node[right]{$u_D$};
			\draw[thick,->] (0,-3) -- (0,3) node[above]{$F(u_D)$};
			
			\draw[domain=0:2.5,name path=exp] plot (\x,{exp(0.7*\x)-3}) node [right] {$F(u_D)$};
			
			\path [name intersections={of=exp and xaxis,by=intersect}];
			\draw [<-, thick, >=stealth] (intersect)  -- ++(0.5,0.5) node [right] {Lösungspunkt};
			\draw (0,-2)node [left] {$-I_0$};		
		\end{scope}
	
	\end{tikzpicture}
\end{center}

\subsubsection{Intervallhalbierung}
\begin{enumerate}
\item Startintervall $[a^{(0)}, b^{(0)}] = [a, b]$, $k = 0$
\item Intervallmitte $m = \frac{a^{(k)} \cdot b^{(k)}}{2}$
\item $[a^{k + 1)}, b^{(k + 1)}] = \begin{cases} [m, b^{(k)}] & \text{fuer\ } F(m) \cdot F(a^{(k)}) > 0 \\ [a^{(k)}, m] & \text{fuer\ } F(m) \cdot F(a^{(k)}) < 0 \end{cases}$
\item Falls $\abs{a^{k + 1)} - b^{k + 1)}} > \epsilon$ dann gehe zu Schritt 2, $k = k + 1$
\end{enumerate}

\begin{center}
\begin{tikzpicture}
\coordinate (a) at (0,0);
\coordinate (b) at (5,0);

\draw [thick] (a) node [left] {$a$} -- (b) node[right] {$b$};
\draw[thick] (a) -- ++(0,-1) node[left] {$f(a)$} [out=10,in=-115] to (1.5625,0) node[above] {\color{red}{$\xi$}} [out=65,in=185] to (5,3);
\draw[thick] (b) -- ++(0,3) node[right] {$f(b)$};

\draw[thick,dashed] ($(a)!.25!(b)$) -- ++(0,-.45);
\draw[thick,dashed] ($(a)!.375!(b)$) -- ++(0,.6);
\draw[thick,dashed] ($(a)!.5!(b)$) -- ++(0,1.6);

\draw[thick] (0,-2) node [left] {$a_0$} -- (5,-2) node[right] {$b_0$};
\draw[thick] (0,-1.9) -- (0,-2.1); 
\draw[thick] (5,-1.9) -- (5,-2.1);
\draw[thick] (2.5,-1.9) node[above] {\color{red}{$x_1$}} -- (2.5,-2.1); 
\draw[thick] (0,-3) node [left] {$a_1$} -- (2.5,-3) node[right] {$b_1$};
\draw[thick] (0,-2.9) -- (0,-3.1); 
\draw[thick] (2.5,-2.9) -- (2.5,-3.1);
\draw[thick] (1.25,-2.9) node[above] {\color{red}{$x_2$}} -- (1.25,-3.1);
\draw[thick] (1.25,-4) node [left] {$a_2$} -- (2.5,-4) node[right] {$b_2$};
\draw[thick] (1.25,-3.9) -- (1.25,-4.1); 
\draw[thick] (2.5,-3.9) -- (2.5,-4.1);
\draw[thick] (1.875,-3.9) node[above] {\color{red}{$x_3$}} -- (1.875,-4.1);
\draw[thick] (1.25,-5) node [left] {$a_3$} -- (1.875,-5) node[right] {$b_3$};
\draw[thick] (1.25,-4.9) -- (1.25,-5.1); 
\draw[thick] (1.875,-4.9) -- (1.875,-5.1); 
\end{tikzpicture}
\end{center}

Die Konvergenz eines iterativen Verfahrens wird mithilfe der Parameter $L$, dem Konvergenzfaktor, und $p$, der Konvergenzordnung, beschrieben.
\[\Delta^{(k + 1)} = \frac{1}{2} \Delta^{(k)} = (\frac{1}{2})^{k + 1} \Delta^{(0)}\]
\[\epsilon = x - x^\star\]
\[\abs{\epsilon^{(k + 1)}} \le L \abs{\epsilon^{(k)}}^p\]
Die Intervallhalbierung liegt bei $p = 1$, also linearer Konvergenz, und $L = 1/2$. Die Anzahl der Iterationsschritte bis zu einem Restfehler $\Delta^R$ ist damit
\[\kappa = \ceil{\ld \left( \frac{\Delta^{(0)}}{\Delta^R}\right)}\]

\begin{center}
	\begin{tikzpicture}
		\draw[thick,->] (0,0) -- (4,0) node[right]{$x$};
		\draw[thick,->] (0,0) -- (0,3) node[above]{$F(x)$};
		
		\draw (0.8, -0.2) node [below] {$x^{(k+2)}$} -- (2.8, 1) node [right] {$\bar{F}^{(k+1)}(x)$};
		\draw (1.2, -0.2) node [below] {$x^{(k+1)}$} -- (3, 2) node [right] {$\bar{F}^{(k)}(x)$};
		
		\draw [dashed] (2, 0) node [below] {$x^{(k)}$} -- ++(0,1);	
	\end{tikzpicture}
\end{center}

\subsubsection{Newton-Raphson-Verfahren} Die Idee hinter dem Newton-Raphson-Verfahren ist eine Taylorreihe erster Ordnung:
\[F(x) = F(x^{(k)}) + F'(x^{(k)}) \cdot (x - x^{(k)}) + F''(x^{(k)}) \cdot \frac{(x - x^{(k)})^2}{2} + \dots\]
\[x^{(k + 1)} = x^{(k)} - \frac{F(x^{(k)})}{F'(x^{(k)})}\]

Das Konvergenzverhalten des Newton-Raphson-Verfahrens erhält man aus einer Taylorreihe:
\[x^{(k + 1)} = x^{(k)} - \frac{F(x^\star) + F'(x^\star)(x^{(k)} - x^\star) + \frac{1}{2} F''(x^\star) \cdot {\epsilon^{k}}^2 + ... + \frac{1}{n!} F^{(n)}(x^\star) \cdot {\epsilon^{k}}^n}{F'(x^\star) + F''(x^\star) \cdot \epsilon^{(k)} + \frac{1}{(n - 1)!} \cdot F^{(n)} {\epsilon^{(k)}}^{n - 1}}\]
\[\epsilon^{(k + 1)} = \epsilon^{(k)} \left( 1 - \frac{F' + \frac{1}{2} F'' \cdot \epsilon^{(k)} + \dots}{F' + F'' \cdot \epsilon^{(k)} + \dots} \right) \approx \epsilon^{(k)} \left( \left(1 + \frac{1}{2} \epsilon^{(k)} \frac{F''}{F'} \right) \cdot \left( 1 - \frac{1}{2} \epsilon^{(k)} \frac{F''}{F'} \right) \right)\]
\[\epsilon^{(k + 1)} \approx \epsilon^{(k)} \left( 1 - \left( 1 + \frac{1}{2} \cdot \epsilon^{(k)} \frac{F''}{F'} - \epsilon^{(k)} \frac{F''}{F'}\right)\right) = \frac{1}{2} \cdot {\epsilon^{(k)}}^2 \frac{F''}{F'}\]

Probleme beim Newton-Raphson-Verfahren treten bei n-fachen Nullstellen ($F'(x^\star) = \dots = F^{(n - 1)}(x^\star) = 0)$, $F^{(n)}(x^\star) \ne 0$) auf.
\[\epsilon^{(k + 1)} \approx \epsilon^{(k)} \cdot \left( 1 - \frac{(n - 1)!}{n!}\right) - \epsilon^{(k)} \cdot \left( 1 - \frac{1}{n}\right) = \epsilon^{(k)} \frac{n - 1}{n}\]
Somit erhält man nur mehr lineare Konvergenz.
z.B.: $F(x) = e^x - 1 \Rightarrow x^{(k + 1)} = x^{(k)} - \frac{e^{x^{(k)}} - 1}{e^{x^{(k)}}}$

\paragraph{Variation: Sekanten-Methode}
\[F'(x^{(k)}) = \lim_{x \rightarrow x^{(k)}} \frac{F(x) - F(x^{(k)})}{x - x^{(k)}}\]
\[F'(x^{(k)}) = \frac{F(x^{(k - 1)}) - F(x^{(k)})}{x^{(k - 1)} - x^{(k)}}\]
Eingesetzt in die Newton-Gleichung erhält man somit
\[x^{(k + 1)} = x^{(k)} - \frac{F(x^{(k)}) \cdot (x^{(k)} - x^{(k - 1)})}{F(x^{(k)}) - F(x^{(k - 1)})}\]
Dadurch ist pro Schritt nur eine Funktionsauswertung nötig.


\subsection{Fixpunktiteration}
Fixpunkt $x^*$ einer Funktion $g(x)\cdot x^* = g(x^*)$.\\
Nullstellenproblem und Fixpunktproblem können ineinander überführt werden., z.B.
\[g(x) = x - F(x)\]

\subsubsection{Fixpunkttheorem}
\begin{enumerate}
\item Sei $g(x)$ stetig auf $[a,b]$, sowie $g(x)\in [a,b]$ für alle $x\in [a,b]$.\\
Dann hat $g(x)$ mindestens einen Fixpunkt auf $[a,b]$
\item Existiere zusaätzlich $g'(x) auf (a,b)$, sowie eine positive Konstante $k<1$ mit $\abs{g'(x)}\leq k$, für alle $x\in (a,b)$
\begin{itemize}
\item Dann hat $g(x)$ exakt einen Fixpunkt auf $[a,b]$
\item Gilt $0<k<1$, dann konvergiert für jedes $x^{(0)}\in [a,b]$ die Folge 
\[x^{(n)} = g(x^{(n-1)})\quad,n\geq 1\]
zum eindeutigen Fixpunkt $x^*\in [0,b]$.
\end{itemize}
\end{enumerate}

\subsubsection{Konvergenzverhalten iterativer Verfahren}
\begin{align*}
x^{(k+1)} &= g(x^{(k)}) & \text{Iterationsvorschrift}\\
x^* &= g(x^*) & \text{Fixpunkt}
\end{align*}

% Figure
	
	% Verzeichnisse
	\cleardoublepage
	\pagenumbering{roman}
	% Abbildungs- und Tabellenverzeichnis
	%\setcounter{lofdepth}{2}
	\listoffigures
	%\listoftables
	
\end{document}