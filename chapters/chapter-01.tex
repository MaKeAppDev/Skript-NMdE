\chapter{Einführung}
\label{chap:einfuehrung}

\section{Zusammenfassung}

\subsection{Überblick}

\subsection{Vorgehen}

\section{Rückblick}

\subsection{Lineares Gleichungssystem}

\begin{align*}
a_{11}\,x_1 + \ldots + a_{1n}\,x_n &= b_1\\
a_{21}\,x_1 + \ldots + a_{2n}\,x_n &= b_2\\
\vdots\\
a_{m1}\,x_1 + \ldots + a_{mn}\,x_n &= b_m\\
\ma{A} \cdot \vec{x} &= \vec{b}\\
\ma{A} \in R^{m\times n},\ \vec{x} \in R^n,\ \vec{b} \in R^m
\end{align*}

\subsection{Matrizen}
\subsubsection{Elementare Matrix-Operationen}


\textbf{Addition:}
\begin{align*}
\ma{A}_{<m\times n>} + \ma{B}_{<m\times n>} &= \ma{C}_{<m\times n>}\\
c_{\mu\nu} &= a_{\mu\nu} + b_{\mu\nu}
\end{align*}

\textbf{Multiplikation mit Skalar:}
\begin{align*}
k \cdot \ma{A} = \ma{A} \cdot k = B\\
b_{\mu\nu} = k \cdot a_{\mu\nu}
\end{align*}

\textbf{Matrix-Multiplikation:}
\begin{align*}
\ma{A}_{<m\times n>} \cdot \ma{B}_{<n\times l>} &= \ma{C}_{<m\times l>}\\
c_{\mu\nu} &= \sum_{\nu=1}^{n} a_{\mu\nu} \cdot b_{\mu\lambda};\quad \mu=1\ldots m,\ \lambda=1\ldots l
\end{align*}

\textbf{Multiplikation Matrix-Vektor:}
\begin{align*}
\text{Spezialfall der Matrix-Multiplikation: } <n\times 1> \text{ bzw. } <1\times n>
\end{align*}

\textbf{Auffassen als Linearkombination der Spalten der Matrix:}
\begin{align*}
\ma{A} \cdot \vec{x} &= \vec{b}\\
\begin{bmatrix}\vec{a}_1 & \vec{a}_2 & \ldots & \vec{a}_n\end{bmatrix} \cdot \vec{x} &= \vec{b}\\
\vec{a}_1\,\vec{x} + \ldots + \vec{a}_n\,\vec{x}_n &= \vec{b}
\end{align*}

\textbf{Vektoren:}
\begin{align*}
\text{Spaltenvektor:} \quad & \vec{a}_{<n\times 1>}: \vec{a}\\
\text{Zeilenvektor:} \quad & \vec{b}_{<1\times m>}^T: \vec{b}^T
\end{align*}

\subsubsection{Vektormultiplikation}
\begin{align*}
\vec{b}^T\cdot \vec{a} &= c \quad \text{(Skalarprodukt $m=n$!)}\\
\vec{a} \cdot \vec{b}^T &= \ma{C} \quad \text{(Vektorprodukt, dyadisches Produkt)}
\end{align*}

\textbf{Matrix als Kombination von Zeilen- und Spaltenvektoren:}
\begin{align*}
\ma{A}_{<m\times n>} &= \begin{bmatrix} \vec{a}_1^T \\ \vec{a}_2^T \\ \vdots \\ \vec{a}_m^T \end{bmatrix}\\
\ma{B}_{<n\times l>} &= \begin{bmatrix} \vec{b}_1 & \vec{b}_2 & \ldots & \vec{b}_l \end{bmatrix}\\
\ma{A} \cdot \ma{B} &= \ma{C}\\
\vec{c}_{\mu\lambda} &= \vec{a}_\mu^T \cdot \vec{b}_\lambda
\end{align*}

\textbf{Rechenregeln:}
\begin{align*}
\left(\ma{A}\cdot \ma{B}\right)\cdot \ma{C} &= \ma{A} \cdot \left(\ma{B} \cdot \ma{C}\right) \quad \text{Assoziativität}\\
\ma{A} \cdot \left(\ma{B} + \ma{C}\right) &= \ma{A}\,\ma{B} + \ma{A}\,\ma{B} \quad \text{Distributivität}\\
\ma{A} \cdot \ma{B} &\neq \ma{B} \cdot \ma{A} \quad \text{Kumutativität gilt im Allgemeinen nicht}
\end{align*}

\textbf{Diagonalmatrix:}
\begin{align*}
\ma{D} &= \text{diag}\begin{bmatrix}d_1 & d_2 & \ldots & d_n\end{bmatrix} = \begin{bmatrix}d_1 & \ldots & 0 \\ 0 & \ldots & d_n\end{bmatrix}\\
\ma{D}_1 \cdot \ma{D}_2 &= \ma{D}_2 \cdot \ma{D}_1
\end{align*}

\textbf{Einheitsmatrix:}
\begin{align*}
\ma{I} = \ma{E} = \ma{1} = \text{diag}\begin{bmatrix}1 & \ldots & 1\end{bmatrix}\\
\ma{A} \cdot \ma{I} = \ma{I} \cdot \ma{A}\\
\ma{I}^n = \ma{I}\\
\ma{I}^{-1} = \ma{I}
\end{align*}