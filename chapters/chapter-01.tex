\chapter{Einführung}
\label{chap:einfuehrung}

\section{Zusammenfassung}

\subsection{Überblick}
\begin{center}
\begin{tikzpicture}[very thick,
		node distance=2.5cm,on grid,>=stealth',
		block/.style={rectangle,draw,rounded corners=1mm,minimum height=1cm,minimum width=2cm}]

\node [block] (top)						{Mathematik};
\node [block] (down3) [below=of top]	{Analysis} edge [-] (top);
\node [block] (down2) [left=of down3]	{Algebra} edge [-] (top);
\node [block] (down1) [left=of down2]	{\begin{tabular}{c}Logik und\\ Mengenlehre\end{tabular}} edge [-] (top);
\node [block] (down4) [right=of down3]	{Stochastik} edge [-] (top);
\node [block] (down5) [right=of down4]	{Numerik} edge [-] (top);
\end{tikzpicture}
\end{center}

\textbf{Definitionsversuch:} Entwicklung und mathematisches Verständnis von numerischen Algorithmen, als von Rechenmethoden zur zahlenmäßigen Lösung mathematischer Probleme.

\textbf{Zusammenspiel mit Informatik:}
\begin{center}
\begin{tikzpicture}[very thick,node distance=1.5cm,on grid,>=stealth']
\node (top)	{Numerische Mathematik};
\node (mid) [below=of top] {\begin{tabular}{l}Scientific Computing\\ Wissenschaftliches Rechnen\end{tabular}} edge [<-] (top);
\node (down) [below=of mid]	{Informatik} edge [<-] (mid);
\node [node distance=7cm, right=of mid] {\begin{tabular}{c}Logik und\\ Mengenlehre\end{tabular}} edge [<-] node [midway,above] {Computer} (mid);
\end{tikzpicture}
\end{center}

\subsection{Vorgehen}
\begin{center}
\begin{tikzpicture}[very thick,node distance=1.5cm,on grid,>=stealth']
\node (top)	{Anwendungsproblem};
\node (down1) [below=of top] {Primäres mathematisches Problem} edge [<-] node [right] {Mathematisierung} (top);
\node (down2) [below=of down1] {Sekundäre mathematische Probleme} edge [<-] node [right] {Mathematische Umformungen} (down1);
\node (down3) [below=of down2] {Algorithmus/Computerprogramm} edge [<-] node [right] {Programmierung} (down2);
\node (down4) [below=of down3] {"'Lösung"'} edge [<-] node [right] {Rechnung} (down3);
\node [below=of down4] {Lösung} edge [<-] node [right] {Beurteilung, "'Entmathematisierung"'} (down4);
\end{tikzpicture}
\end{center}

\subsection{Aspekte für die Lösung einer mathematischen Aufgabenstellung}
\begin{itemize}
\item Kondition eines Problems (Empfindlichkeit für Störungen)
\item Numerische Lösungsverfahren
\item Stabilität des Lösungsverfahrens (Empfindlichkeit für Störungen)
\item Effizienz des Lösungsverfahrens
\item Genauigkeit der Lösung
\end{itemize}

\section{Wiederholung}

\subsection{Lineares Gleichungssystem}

\begin{align*}
a_{11}\,x_1 + \ldots + a_{1n}\,x_n &= b_1\\
a_{21}\,x_1 + \ldots + a_{2n}\,x_n &= b_2\\
&\vdots\\
a_{m1}\,x_1 + \ldots + a_{mn}\,x_n &= b_m\\
\ma{A} \cdot \vec{x} &= \vec{b}\\
\ma{A} &\in R^{m\times n},\ \vec{x} \in R^n,\ \vec{b} \in R^m
\end{align*}

\subsection{Matrizen}
\subsubsection{Elementare Matrix-Operationen}


\textbf{Addition:}
\begin{align*}
\ma{A}_{<m\times n>} + \ma{B}_{<m\times n>} &= \ma{C}_{<m\times n>}\\
c_{\mu\nu} &= a_{\mu\nu} + b_{\mu\nu}
\end{align*}

\textbf{Multiplikation mit Skalar:}
\begin{align*}
k \cdot \ma{A} = \ma{A} \cdot k = B\\
b_{\mu\nu} = k \cdot a_{\mu\nu}
\end{align*}

\textbf{Matrix-Multiplikation:}
\begin{align*}
\ma{A}_{<m\times n>} \cdot \ma{B}_{<n\times l>} &= \ma{C}_{<m\times l>}\\
c_{\mu\nu} &= \sum_{\nu=1}^{n} a_{\mu\nu} \cdot b_{\mu\lambda};\quad \mu=1\ldots m,\ \lambda=1\ldots l
\end{align*}

\textbf{Multiplikation Matrix-Vektor:}
\begin{align*}
\text{Spezialfall der Matrix-Multiplikation: } <n\times 1> \text{ bzw. } <1\times n>
\end{align*}

\textbf{Auffassen als Linearkombination der Spalten der Matrix:}
\begin{align*}
\ma{A} \cdot \vec{x} &= \vec{b}\\
\begin{bmatrix}\vec{a}_1 & \vec{a}_2 & \ldots & \vec{a}_n\end{bmatrix} \cdot \vec{x} &= \vec{b}\\
\vec{a}_1\,\vec{x} + \ldots + \vec{a}_n\,\vec{x}_n &= \vec{b}
\end{align*}

\textbf{Vektoren:}
\begin{align*}
\text{Spaltenvektor:} \quad & \vec{a}_{<n\times 1>}: \vec{a}\\
\text{Zeilenvektor:} \quad & \vec{b}_{<1\times m>}^T: \vec{b}^T
\end{align*}

\subsubsection{Vektormultiplikation}
\begin{align*}
\vec{b}^T\cdot \vec{a} &= c \quad \text{(Skalarprodukt $m=n$!)}\\
\vec{a} \cdot \vec{b}^T &= \ma{C} \quad \text{(Vektorprodukt, dyadisches Produkt)}
\end{align*}

\textbf{Matrix als Kombination von Zeilen- und Spaltenvektoren:}
\begin{align*}
\ma{A}_{<m\times n>} &= \begin{bmatrix} \vec{a}_1^T \\ \vec{a}_2^T \\ \vdots \\ \vec{a}_m^T \end{bmatrix}\\
\ma{B}_{<n\times l>} &= \begin{bmatrix} \vec{b}_1 & \vec{b}_2 & \ldots & \vec{b}_l \end{bmatrix}\\
\ma{A} \cdot \ma{B} &= \ma{C}\\
\vec{c}_{\mu\lambda} &= \vec{a}_\mu^T \cdot \vec{b}_\lambda
\end{align*}

\textbf{Rechenregeln:}
\begin{align*}
\left(\ma{A}\cdot \ma{B}\right)\cdot \ma{C} &= \ma{A} \cdot \left(\ma{B} \cdot \ma{C}\right) \quad \text{Assoziativität}\\
\ma{A} \cdot \left(\ma{B} + \ma{C}\right) &= \ma{A}\,\ma{B} + \ma{A}\,\ma{B} \quad \text{Distributivität}\\
\ma{A} \cdot \ma{B} &\neq \ma{B} \cdot \ma{A} \quad \text{Kumutativität gilt im Allgemeinen nicht}
\end{align*}

\textbf{Diagonalmatrix:}
\begin{align*}
\ma{D} &= \text{diag}\begin{bmatrix}d_1 & d_2 & \ldots & d_n\end{bmatrix} = \begin{bmatrix}d_1 & \ldots & 0 \\ 0 & \ldots & d_n\end{bmatrix}\\
\ma{D}_1 \cdot \ma{D}_2 &= \ma{D}_2 \cdot \ma{D}_1
\end{align*}

\textbf{Einheitsmatrix:}
\begin{align*}
\ma{I} = \ma{E} = \ma{1} = \text{diag}\begin{bmatrix}1 & \ldots & 1\end{bmatrix}\\
\ma{A} \cdot \ma{I} = \ma{I} \cdot \ma{A}\\
\ma{I}^n = \ma{I}\\
\ma{I}^{-1} = \ma{I}
\end{align*}

\textbf{Transponierte Matrix:}
\begin{align*}
\ma{A}_{<m\times n>}^T &= \ma{B}_{<n\times m>},\ b_{\nu\mu} = a_{\mu\nu}\\
\left(\ma{A} \cdot \ma{B}\right)^T &= \ma{B}^T \cdot \ma{A}^T\\
\ma{A} &= \ma{A}^T \quad \Rightarrow \text{ symmetrische Matrix}
\end{align*}

\textbf{Inverse Matrix:}
\begin{align*}
\ma{A} &\in R^{n\times n}\\
\ma{A}^{-1} \cdot \ma{A} &= \ma{A} \cdot \ma{A}^{-1} = \ma{I}\\
\text{$\ma{A}^{-1}$ existiert nur für nicht singuläre $\ma{A}$}&\text{ und ist eindeutig.}\\
\ma{A} = \text{diag}\begin{bmatrix}d_1 & \ldots & d_n\end{bmatrix} & \Rightarrow \ma{A}^{-1} = \text{diag}\begin{bmatrix}\frac{1}{d_1} & \ldots & \frac{1}{d_n}\end{bmatrix}\\
\left(\ma{A} \cdot \ma{B}\right) &= \ma{B}^{-1} \cdot \ma{A}^{-1}
\end{align*}

\section{Schaltungsanalyse}
\begin{center}
\begin{circuitikz}
\draw
	(0,4)
	to [R=$Y_1$, i=\textcolor{green!70!blue}{$i_1$},-o] (4,4) node[above] {1}
	to [R=$Y_3$, i=\textcolor{green!70!blue}{$i_3$},v>=\textcolor{blue}{$u_3$},-o] (8,4) node[above] {3}
	to [R=$Y_5$, i<^=\textcolor{green!70!blue}{$i_5$},v=\textcolor{blue}{$u_5$},o-o] (12,4) node[above] {2} 
	to [short,i<=\textcolor{green!70!blue}{$i_2$}] (12,3)
	to [R=$Y_2$,v=\textcolor{blue}{$u_2$},*-*] (12,1)
	to (13.5,1)
	to [I,i_=$I_{02}$] (13.5,3) to (12,3)
	(12,1) to (12,0) to (8,0) node[ground] {}
	to [R=$Y_4$, i=\textcolor{green!70!blue}{$i_4$},v>=\textcolor{blue}{$u_4$},o-] (8,4) 
	(8,0) +(0.2,0) node[above] {0} to (0,0)
	to [V=$U_{01}$] (0,4)
	(4,4) 
	to [open,v=\textcolor{blue}{$u_1$}] (0,0);

\draw[color=red](8,0)
	to [open,v^>=\textcolor{red}{$u_{n1}$}] (4,4)
	(9,0)
	to [open,v>=\textcolor{red}{$u_{n3}$}] (9,4)
	(11,0)
	to [open,v^>=\textcolor{red}{$u_{n2}$}] (11,4);

\end{circuitikz}
\end{center}
\begin{tabular}{llll}
\textcolor{green!70!blue}{$\vec{i}_{<k>}$} & \textcolor{green!70!blue}{Kantenstromvektor} & \textcolor{blue}{$\vec{u}_{<k>}$} & \textcolor{blue}{Kantenspannungsvektor}\\
$\vec{i}_{0<k>}$ & Kantenquellenstromvektor & $\vec{u}_{0<k>}$ & Kantenquellenspannungsvektor\\
$\vec{i}_{n<k>}$ & Knotenquellenstromvektor & \textcolor{red}{$\vec{u}_{n<k>}$} & \textcolor{red}{Knotenquellenspannungsvektor}\\
\\
$\ma{A}_{<n\times k>}$ & Kontenmatrix, Knoteninzidenzmatrix & &\\
$\ma{Y}_{<k\times k>}$ & Kantenadmittanzmatrix & &\\
$\ma{Y}_{n<n\times n>}$ & Kontenadmittanzmatrix & &\\
\end{tabular}

\subsection{Gerichteter Graph der Schaltung}

\subsection{Inzidenz Matrix}
\begin{align*}
\ma{A}_{<m\times n>} = \begin{bmatrix}
1 & & & -1\\
 & 1 & & -1\\
-1 & & 1 & \\
 & & 1 & -1\\
 & -1 & 1 & 
\end{bmatrix}
\end{align*}
Summen der Spaltenvektoren $= \vec{0}$.\\
$\Rightarrow \ma{A}$ hat linear abhängige Spalten.

Rang der Matrix $\ma{A}$: $r = \rang{\ma{A}} = 3 = n-1$.\\
Dimension des Nullraums: Zahl der Spalten $-\ r = 1$ .\\
Vektor im Nullraum von $\ma{A}$:
\begin{align*}
\ma{A}\cdot\vec{u}=\vec{0}\Rightarrow\vec{u}\in ;\quad\vec{u}=\begin{bmatrix}1\\1\\1\\1\end{bmatrix}
\end{align*}

\subsection{Laplace Matrix}
\begin{align*}
\ma{A}^T\cdot\ma{A} = \begin{bmatrix}
1 & 0 & -1 & 0 & 0\\ 
0 & 1 & 0 & 0 & -1 \\ 
0 & 0 & 1 & 1 & 1 \\ 
-1 & -1 & 0 & -1 & 0
\end{bmatrix}\cdot\begin{bmatrix}
1 & 0 & 0 & -1 \\ 
0 & 1 & 0 & -1 \\ 
-1 & 0 & 1 & 0 \\ 
0 & 0 & 1 & -1 \\ 
0 & -1 & 1 & 0
\end{bmatrix} = \begin{bmatrix}
2 & 0 & -1 & -1 \\ 
0 & 2 & -1 & -1 \\ 
-1 & -1 & 3 & -1 \\ 
-1 & -1 & 3 & -1
\end{bmatrix}  
\end{align*}
\begin{itemize}
\item singulär
\item $r = 3$
\item symmetrisch
\end{itemize}
\begin{align*}
\ma{A}^T\cdot\ma{A} = \underset{\text{Grad (degree)}}{\ma{D}} - \underset{Adjazenz}{\ma{W}} = \begin{bmatrix}
2 & 0 & 0 & 0 \\ 
0 & 2 & 0 & 0 \\ 
0 & 0 & 3 & 0 \\ 
0 & 0 & 0 & 3
\end{bmatrix} - \begin{bmatrix}
0 & 0 & 1 & 1 \\ 
0 & 0 & 1 & 1 \\ 
1 & 1 & 0 & 1 \\ 
1 & 1 & 1 & 0
\end{bmatrix} 
\end{align*}

\subsection{Kirchhoffsches Stromgesetz (KCL)}

\begin{tabular}{lll}
$\ma{A}^T\cdot \vec{w} = \vec{0}$ & (keine Stromquellen) & $\vec{w}$: Kantenströme\\
$\ma{A}^T\cdot \vec{w} = \vec{f}$ & (mit Stromquellen) & $\vec{f}$: Stromquellen
\end{tabular}

\subsection{Ohmsches Gesetz}
\begin{tabular}{ll}
$\vec{w} = \ma{C} \cdot \vec{e}$ & $\ma{C}$: Diagonalmatrix der Kantenleitwerte\\
 & $\vec{e}$: Kantenspannungen
\end{tabular}

\subsection{Kirchhoffsches Spannungsgesetz (KVL)}
\begin{tabular}{ll}
$\vec{e} = \vec{b} - \ma{A} \cdot \vec{u}$ & $\vec{u}$: Kantenspannungen - GESUCHT\\
 & $\vec{b}$: Spannungsquellen
\end{tabular}

\begin{align*}
\ma{A}^T \cdot \vec{w} = \vec{f}\\
\ma{A}^T \cdot \ma{C} \cdot \vec{e} = \vec{f}\\
\ma{A}^T \cdot \ma{C} \left( \vec{b} - \ma{A}\cdot\vec{u}\right) = \vec{f}\\
\underset{\text{Systemmatrix, gewichtete Laplace Matrix}}{\ma{A}^T \cdot \ma{C} \cdot \ma{A}} \cal \vec{u} = \ma{A}^T \cdot \ma{C} \cdot \vec{b} \cdot \vec{f}\\
\ma{Y}_{<n\times n>} \cdot \vec{u}_{<n>} = \vec{d}_{<n>}
\end{align*}

singulär $\Rightarrow$ nicht invertierbar\\
$\Rightarrow$ keine Lösung für Gleichungssystem\\
Abhilfe: Festlegung eines Bezugspunktes, z.B. $u_0 = 0$\\
\ldots

\subsection{Alternative Darstellung}
\begin{align*}
\ma{C}^{-1} \cdot \vec{w} + \ma{A} \cdot \vec{u} &= \vec{b}\\
\ma{A}^T \cdot \vec{w} = \vec{f}\\
\begin{bmatrix}
\ma{C}^{-1} & \ma{A} \\ \ma{A}^T & \ma{0}\end{bmatrix} \cdot \begin{bmatrix}\vec{w} \\ \vec{w}\end{bmatrix} = \begin{bmatrix}\vec{b}\\\vec{f}\end{bmatrix}
\end{align*}

\ldots

\subsection{Lösung eines linearen Gleichungssystems}
\begin{align*}
\ma{A} \cdot \vec{x} = \vec{b}, \qquad \ma{A} \in R^{n\times n}; \quad \vec{b}, \vec{x} \in R^n, \quad \ma{A} \neq 0\\
\text{Theoretische: } \vec{x} = \ma{A}^{-1} \cdot \vec{b}\\
\text{Alternativ: Lösung durch "`Division"'-Eliminierung} 
\end{align*}

\subsection{Methode zur Bestimmung von $A^{-1}$}
\begin{align*}
\ma{A}\cdot\ma{A}^{-1} &= \ma{I}\\
\text{Sei }\ma{X} &= \ma{A}{^-1}\\
\ma{A}\cdot\ma{X} &= \ma{I}\\
\ma{A}\cdot\begin{bmatrix}
\vec{x}_1 & \vec{x}_2 & \ldots & \vec{x}_n
\end{bmatrix} &= \begin{bmatrix}
\vec{l}_1 & \vec{l}_2 & \ldots & \vec{l}_n
\end{bmatrix};\quad ...
\end{align*}
$\ma{A}\cdot\vec{x}_1 = \vec{l}_1;\quad\ma{A}\cdot\vec{x}_2 = \vec{l}_2;\quad\ldots;\ma{A}\cdot\vec{x}_n = \vec{l}_n;$\\
Nur rechte Seite ändert sich; $\ma{A} = \ma{L}\cdot\ma{U}$ nur 1 mal erforderlich.\\
Aufwand $n\cdot 2\cdot\frac{1}{2}(n^2 + n) = n^3 + n^2 \rightarrow O(n^3)$ für Substitution.

Probleme:
\begin{itemize}
\item Pivot $p = 0$
\item Pivot $p$ "'sehr klein"'
\end{itemize}
Abhilfe: "'Pivotisierung"'- Zeilen-/Spaltentausch
\begin{itemize}
\item "'partial pivoting"': Wähle Zeile mit betragsmäßig größtem Element in Pivotspalte
\item "'complete pivoting"': Zeilen- und Spaltentausch
\item Pivotisierung auch zur Aufwandreduktion
\end{itemize}
